\section{Research methodology}

In order to design a stock market simulator, the first step is to study the relevant aspects of real stock markets, therefore we studied the literature on market microstructure of electronic, limit order book stock markets.

Similarly, in order to realistically model the behaviour of a stock market as a whole, we turned to analyse existing Agent-Based Artificial Stock Markets. We specifically focused on the types of behaviours that are modelled in order to choose an appropriate implementation for the Noise Traders.

The results of the analysis of stock markets and Agent-Based Artificial Stock Markets enabled us to design a flexible and modular framework for the simulator, that effectively mirrors the features of real stock markets. The design and implementation phase followed an iterative approach, with short (2 weeks) iterations, and work progressed on an issue-by-issue basis (features were submitted to an issue tracker and then closed when implemented).

After completing the design and implementation phases, the simulator was evaluated by performing two experiments that correspond to the two null hypotheses. By introducing the simulator and demonstrating its usefulness in discovering errors in Trading Algorithms we aimed to provide a methodological advancement that will be a useful complement to traditional ways of testing Trading Algorithms.
