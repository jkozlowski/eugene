\section{Agent-Based Artificial Stock Markets}
\label{Chapters/Background/Agent-Based-Modelling}

In this section we summarise several artificial stock markets from the available literature. We provide a model of zero-intelligence agents that will be implemented by the Noise Traders, that is based on the work by~\citet[chap.~4]{Gilles2006}. Moreover, in order to provide a background material for implementing a VWAP Trading Algorithm for the experiments, we survey available sources, most prominently the work in~\cite{Coggins2006, Kakade2004}. 

\subsection{Agent-Based Modelling}
Agent-Based Modelling is a simulation technique concerned with designing societies of rule-based software agents that interact in particular ways, with a view of assessing the influence of individual (or groups of) agents on the system as a whole. This technique has been successfully applied in many business scenarios, including financial simulations. Among the benefits of agent-based modelling is emergent phenomena: behaviour resulting from complex interactions of many individual entities.

Agent-Based Models that attempt to explain economic processes are branded as Agent-Based Computational Economics. In recent years, studying stock markets using multi-agent based models has become a promising research area due to the fact that this methodology reflects the fundamental nature of a stock market, where the current situation is a result of a complex interaction of actions of many heterogenous investors that have various expectations and different levels of rationality.

We will now turn to review a number of existing Agent-Based Artificial Stock Markets.

\subsection{SantaFe ASM}
The Santa Fe Artificial Stock Market (SFI ASM)~\citep{Lebaron2002, Lebaron99} is a discrete time artificial stock market that consists of a central computational market and a number of intelligent agents. Agents make decisions by attempting to forecast the future returns on the stock using genetic programming and therefore decide between investing in risky stock or leaving their money in the bank, which pays a fixed interest rate.

The simulated market consists of $N$ agents, usually $50-100$, that interact with the market. The stock has the price $p(t)$ per share at time $t$, where $p(t)$ is set endogenously to clear the market (\textit{call market}). Similarly, the stock pays a dividend $d(t+1)$ at the end of time $t$, according to a stochastic process. The money left in the bank pays constant rate of return of $r$ per period. The information available to agents consists of the price, the dividend, total number of bids/asks at each past period, plus some additional variables. 

By using genetic programming, agents can explore a wide range of possible forecasting rules and they have the flexibility to use or disregard certain pieces of available information. The market was able to generate the key stylised facts: weak forecastability, volatility persistence, and higher expected returns.~\cite{Lebaron99}. \marginpar{I should probably add some explanation of stylised facts somewhere above, because I am going to refer to those in all those ASMs} 

\subsection{Genoa ASM}
Similarly to SFI ASM, The Genoa Artificial Stock market (GASM)~\citep{Raberto2001} is an agent-based artificial financial market in which heterogeneous agents exchange cash and stock from an initial fixed endowment, therefore the total amount of cash and shares is constant throughout the duration of the simulation. The price formation process is set at the intersection of the demand and supply curves (\textit{call market}). At each time step, agents place random buy or sell orders, subject to available resources and clustering, for prices that depend on historical volatility.

The market was able to reproduce key stylised facts, i.e. fat tails and volatility clustering, however failed to reproduce others, e.g. the volatility exhibited an exponential decay, as opposed to power law decay.

\subsection{ABSTRACTE}
The ASMs surveyed thus far evolve in discrete time intervals, i.e. at each time period $t$, some fraction of traders submit orders to the market, the orders are cleared, a new market price is announced, the agents update their positions and a new time period $t+1$ starts. This style of simulation is well placed when dealing with call markets, however fails to realistically reflect the nature of continuous trading sessions, that evolves in a highly asynchronous manner thanks to limit order book microstructure.  

The importance of asynchronous simulation, as opposed to discrete-time simulation was demonstrated by~\citet{Sorban2008}, who describes \textit{ABSTRACTE}: Agent-Based Simulation of Trading Roles in an Asynchronous Continuous Trading Environment. \textit{ABSTRACTE} was used to model a market with information asymmetry, where prices are set by a learning market maker; this model extends the work in~\citep{Das2006}, but evolves the simulation in continuous time. They show that moving to continuous, asynchronous time simulation renders different price dynamics and conclude that continuous nature of trading in real stock markets should be explicitly taken into account in agent-based models.

\subsection{Zero-intelligence model of price formation}

Another strong case for turning to asynchronous simulation is presented by~\citet[chap.~4]{Gilles2006}, who studied the primary role played by liquidity dynamics in price formation mechanism. Liquidity is modelled using zero-intelligence agents trading through the asynchronous limit order book, whose order placement and cancellation process, in terms of order type, size and limit price, are designed to come as close to a realistic aggregate order flow. The simulation is implemented on top of \textit{Natural Asynchronous-Time Event-Lead Agent-Based Platform (NatLab)}.  

The simulation consists of $N$ Zero-intelligence agents, initially endowed with a fixed amount of cash and shares (similarly to~\cite{Raberto2001}). The stock does not distribute dividends and cash does not yield interest. There is not a market maker, therefore the price formation mechanism arises through the non-linear interactions between agents demand and supply.

The agents place buy or sell orders with equal probability. The agents can either cancel an existing order with probability $\pi_c$, send a market order ($\pi_m$), or send a limit order ($1-\pi_c-\pi_c$). The limit order price is either in the spread and uniformly distributed between best bid and best ask ($\pi_{in}$), or outside the spread and power law distributed ($\pi_{out}$).

The order sizes of limit orders follow log-normal distribution, and market order sizes match the best counterpart. In order to model the well documented pattern of trading activity that is more dense at the start and end of the day~\cite{Clark1973}, the agents sleep for random times that are drawn from a stretched exponential distribution, that is obtained as a mixture of two exponential distributions with different means (the influence of sleep time on price dynamics is explored in~\cite{Scalas2004}). The various parameters that relate to this model and corresponding values are summarised in \Cref{Table/Zero-Intelligence}.

\begin{table}[htbp]
\begin{center}
\begin{tabular}{ l l l }
\textbf{Description} & \textbf{Parameter} & \textbf{Value} \\
\cmidrule(r){1-3}
Prob. of cancelling an order & $\pi_c$    		& 0.5                \\
Prob. of market order		 & $\pi_m$   		& 0.15               \\
Prob. limit order in spread  & $\pi_{in}$ 		& 0.35               \\ 
Limit price outside spread   & $1+\alpha$ 		& 1.3                \\
Order size					 & $(\mu, \sigma)$	& (4.5, 0.8) shares  \\
Sleeping time				 & $\tau_1, \tau_2$ & (600, 1800) sec.   \\
\end{tabular}
\end{center}
\caption{Summary of Zero-intelligence model of price formation}
\label{Table/Zero-Intelligence}
\end{table}

This simple model of zero-intelligence agents was able to reproduce realistic dynamics in terms of spread and book shape, as well as slight anti-persistence or tick-by-tick returns and a high level of kurtosis and fat tails. 

A very important feature of simulations implemented on top of \textit{NatLab} is full-reproducibility of simulations: the random seeds used by agents to drive their decisions can be exported and reused, hence allowing the simulations to be reproduced exactly. Therefore, sensitivity analysis performed using this simulator allows the modeller to have confidence that a change observed in the output from the simulation is the result of only the controlled change of parameters, not a product of random occurrence.

Zero-intelligence model as described in~\citet[chap.~4]{Gilles2006} will be used to implement the Noise Traders in \textit{Eugene}. We will now turn to survey available literature on implementing VWAP trading.  







   







