\section{Agent-Based Artificial Stock Markets}
\label{Chapters/Background/Agent-Based-Modelling}

In this section we summarise several artificial stock markets from the available literature. We provide a model of zero-intelligence agents that will be implemented by the Noise Traders, that is based on the work by~\citet[chap.~4]{Gilles2006}. Moreover, in order to provide a background material for implementing a VWAP Trading Algorithm for the experiments, we survey available sources, most prominently the work in~\cite{Coggins2006, Kakade2004}. 

\subsection{Agent-Based Modelling}
Agent-Based Modelling is a simulation technique concerned with designing societies of rule-based software agents that interact in particular ways, with a view of assessing the of of individual (or groups of) agents on the system as a whole. This technique has been successfully applied in many business scenarios, including financial simulations. Among the benefits of agent-based modelling is emergent phenomena: behaviour resulting from complex interactions of many individual entities.

\subsection{Agent-Based Computational Economics}
Agent-Based Models that attempt to explain economic processes are branded as Agent-Based Computational Economics. In recent years, studying stock markets using multi-agent based models has become a promising research area due to the fact that this methodology reflects the fundamental nature of a stock market, where the current situation is a result of a complex interaction of actions of many heterogenous investors that have various expectations and different levels of rationality.

\subsection{An overview of Agent-Based Artificial Stock Markets}

\subsubsection*{SantaFe ASM}








