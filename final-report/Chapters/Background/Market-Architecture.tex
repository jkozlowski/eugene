\section{Market microstructure}
\label{Chapters/Background/Market-Architecture}

In this section we describe the model of the market mechanism we will use in the simulator. We describe the market microstructure employed in major stock exchanges during continuous trading sessions, namely model of limit order book, and how it enables market participants to interact asynchronously (for a study of market microstructures in main stock exchanges, see~\cite{Comerton2004}). 

\subsection{Order Types}
Market participants indicate their willingness to trade in the form of trading instructions called \textit{orders}. We will consider only two types of orders: \textit{limit} and \textit{market} orders. A \textit{market order} specifies the instrument to trade, the quantity and the side of the trade (buy or sell). A \textit{limit order} additionally specifies a \textit{limit price}: maximum (buy) or minimum (sell) price that the trader accepts for an order. 

\subsection{The limit order book}
The limit order book is the leading market mechanism used by main stock exchanges during continuous trading. It consists of limit orders, characterised by a limit price, a volume and a time of arrival. The limit orders are sorted by price and time or arrival and stored in two queues: one for bid (buy) orders and one for ask (sell) orders. At a specific instant in time $t$, the order book can be described as~\cite{Gilles2006}: 
\begin{equation*}
\beta_n \leq \ldots \leq \beta_2 \leq \beta_1 < \alpha_1 \leq \alpha_2 \leq \ldots \alpha_m
\end{equation*}
where $\beta_i$ represent bid orders, $\alpha_j$ represent ask orders. The highest bid $\beta_1$ (or best bid) and lowest ask $\alpha_1$ (or best ask) define the spread $\alpha_1 - \beta_1$.

An incoming limit order can either trigger a trade or be stored in the book.    $\beta_1$ will be executed only if the book receives a market sell order, or a limit sell order with an offering price lower than or equal to $\beta_1$. In this case a trade will be executed at the price of $\beta_1$. Analogously for $\alpha_1$.






