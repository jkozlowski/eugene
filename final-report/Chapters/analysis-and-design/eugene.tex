\section{Eugene}

\subsection{Approach}
Agent-Based simulation is a powerful technique that has been successfully applied in many business scenarios, including financial simulations. Among the benefits of agent-based modelling is emergent phenomena: behaviour resulting from complex interactions of many individual entities. 

Therefore, whereas in a real stock market, the current situation is a result of a complex interaction of many market players, it is hypothesised that an interaction of several types of software agents will result in an emergence of realistic market behaviour.

\subsection{Technologies}
\subsubsection{Java Agent DEvelopment Framework}
\texttt{JADE} is a software framework for developing distributed, multi-agent software systems in Java. \texttt{JADE} can be deployed onto a set of containers (nodes) which together form a platform (cluster).

Each platform has a \texttt{Main Container} which holds two special agents:
\begin{itemize}
\item The \texttt{Agent Management System} (AMS) which can create and kill agents, kill containers and shut down the entire platform.
\item The \texttt{Directory Facilitator} (DF) which implements a \texttt{yellow pages} service.
\end{itemize} 

Each agent in \texttt{JADE} operates in a separate thread of control, therefore allowing independent, preemptive behaviour. \texttt{JADE}  has been designed to operate within an \texttt{OSGi} container, therefore making it easier to deploy.

\subsubsection{Open Services Gateway initiative framework}
\texttt{OSGi} is dynamic module and service system platform for Java. Application components (distributed as \texttt{bundles}) can be remotely managed without requiring a reboot of the entire container. A package management system enables developers to package public and private APIs within the same bundle, but only expose public APIs at runtime. The dynamic service system allows the components to discover the addition of new services and act accordingly.

\subsection{Simulation Structure}
\begin{itemize} 
\item Eugene will simulate one day of a continuous auction trading.
\item Eugene will aim to simulate a set of typical markets, according to parameters such as: Volume Curve, Price Volatility, Bid/Ask spread, average order size etc. (list not exhaustive).
\item Eugene will aim to simulate the following set of agents (list is not exhaustive):
\begin{itemize}
\item High Frequency Trading Agent: agent that moves in and out of short-term positions many times each day, to capture trading opportunities that may open up only for fractions of a second.
\item Gaming Agent: agent that is designed to try to trick Trading Algorithms in order to manipulate them for its own gain.
\item Random Agent: agent that will make random decisions to send mid, aggressive or passive orders.
\item Technical Analysis Agent: agent that will make decisions based on past behaviour of the market.
\item etc.
\end{itemize}
\end{itemize}

\subsection{Aims}
\begin{itemize}
\item Analyse existing Agent-Based approaches to Market Simulation to determine a set of possible Agent Types.
\item Implement the Agents.
\item Discover a specific configuration of Agents whose behaviour will simulate each target market, in regards to a set of pre-determined market characteristics (Volume Curve, Price Volatility etc.).
\item Analyse efficacy of the Agent-Based Market Simulation approach for use in System Testing of Trading Algorithms.
\end{itemize}

\subsection{Success Factors}
\begin{itemize}
\item Develop a realistic simulation for at least one target market.
\end{itemize}
