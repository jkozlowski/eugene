\section{Market Agent}
\label{Chapters/Implementation/Market-Agent}
The \texttt{Market Agent} plays the role of the Stock Exchange with a central limit order book. The implementation is very clearly split between the part that maintains the order book, executes orders etc. and the messaging subsystem, in order to maintain clear separation of concerns and improve ease of unit-testing. 

The messaging subsystem implementation is located in the \\\texttt{eugene.market.esma.impl.behaviours} package and consists of implementations of \texttt{jade.core.behaviours.Behaviour}. Generally, all behaviours in this package act as message gateways: they accept messages but, except for messages that fail basic error checks, do not send replies.  

\texttt{OrderServer} accepts messages to create new orders, cancel existing orders and handle the logon, i.e. messages from \texttt{Market Ontology} (\Cref{Chapters/Implementation/Market-Ontology}). Similarly, \texttt{SimulationOntologyServer} deals with messages from \texttt{Simulation Ontology} (\Cref{Chapters/Implementation/Simulation-Ontology}). On the other hand, \texttt{MarketDataServer} deals with sending out trade confirmations to counter parties and market data events.

The second part of the \texttt{Market Agent} is located in the \\ \texttt{eugene.market.esma.impl.execution} package. \texttt{ExecutionEngine} deals with managing the order book, accepting new orders, cancelling existing orders and matching orders. All the various parts of the \texttt{ExecutionEngine} are refactored to separate classes for ease of testing: \texttt{MatchingEngine} implements the matching algorithm and \texttt{InsertionValidator} checks whether a \texttt{Market Order} can be accepted or should be rejected, because there is no liquidity on the book. Every change to the limit order book (inserting a new order, cancelling an existing order and executing an order) triggers an event that is recorded in the \texttt{MarketDataEngine} (\texttt{eugene.market.esma.impl.execution.data} package).

The points of synchronisation between the messaging subsystem and the execution subsystem are maintained in two classes: \texttt{Repository} (\texttt{eugene.market.esma.impl} package) and \texttt{MarketDataEngine}.  \texttt{Repository} maintains the mapping between current active orders and the owner traders; whenever an order is accepted by the \texttt{OrderServer} it is recorded in the \texttt{Repository}. Similarly, \texttt{MarketDataServer} retrieves events from the \texttt{MarketDataEngine} and sends them out.

In order to achieve high incoming order rate (requirement \texttt{N01}), the \texttt{OrderServer} and \texttt{SimulationOntologyServer} operate in a different thread than \texttt{MarketDataServer}. Therefore, both \texttt{Repository} and \texttt{MarketDataEngine} need to be thread-safe (but not the rest of the classes, as they are not shared by different threads). Due to a low number of synchronisation points and clear separation between mutable and immutable state of the order book (\Cref{Chapters/Implementation/Order-Book}), going from single-threaded to multi-threaded design of the \texttt{Market Agent} was relatively straightforward.
