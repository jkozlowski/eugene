
\begin{itemize}
\item Describe the design of what you have created.
\item Start with the application architecture, giving its overall structure and the components that make up that structure.
\item Give a description of the design of each of the the components that make up the architecture.
\item Include the database or storage representation.
\item Provide implementation details as necessary.
As with other chapters, the structure and contents of this chapter will depend on the nature of your project, so the list above is only a suggestion not a fixed requirement.
\end{itemize}


Find an ordering and form of words so that the design is clear, focusing on the interesting design decisions. For example, what were the alternatiect one particular solution? You have a limited number of pages so be selective about details. Also remember that someone (your examiners!) has to read this so don’t overwhelm them with intricate descriptions of everything that only you can follow – but do make sure the key details of the solution are in place. Use appropriate terminology and demonstrate that you have a good understanding of the Computer Science principles involved.

You can use diagrams and screen shots to help explain the design but don’t overuse them. Diagrams and screen shots should add information, not duplicate what is written in the text, and definitely avoid page after page of diagrams as this will disrupt the flow of your text. Where relevant, UML diagrams can certainly be used but, again, don’t flood the chapter with diagrams. Additional diagrams can always be included in an appendix section.

It may be useful to include sections of code to highlight how a particular algorithm is implemented or how an interesting programming problem was solved. However, avoid lengthy sections of code, as they can disrupt the flow of the text. Also make sure that your code fragments are readable, easy to follow and properly laid out. It may be better to use pseudo-code rather than actual code, especially when describing an algorithm. If you need to make use of longer sections of code, you can put the code in the appendix and reference it from the text.
An alternative way to organise the content of both this chapter and the preceding one, suitable for some projects, is to have a sequence of chapters for each major iteration of the project. This allows the progression of the project to be shown, with each iteration building on the last.
This is a core chapter in your report and will usually be quite substantial, 10 pages or more.

\section{Things to cover}
\begin{itemize}
\item Modules.
\item Logging.
\item Ontologies.
\end{itemize}

\section{Things to draw attention to}
\begin{itemize}
\item The principle of defensive programming.
\item Separation of concerns with extensive use of programming to an interface.
\item Information hiding with use of private packages (OSGi) that contain implementations;
\item Hiding implementations behind factories.
\item Modularised design.
\item Fluent APIs.
\item Extensive unit and integration testing.
\item jade-unit.
\item mocking.
\item Immutable classes.
\item Careful consideration of threading issues and hence multithreaded design.
\item Highlight the fact that I needed to work around the problems of testing, etc.
\item Focus on how problems were anticipated, so they were planned.

\end{itemize}