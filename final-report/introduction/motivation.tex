\section{Motivation}

\todo{Add references to all the parenthesis}

The design of a software system gradually evolves over time, as the understanding of the problem domain improves and the business requirements change. The cost of change of a design of a software system \todo{Too many ofs} is not trivial, however it is comparably lower than in other Engineering Disciplines and can be controlled by adhering to principles of good design (Design Patterns). This propensity of software encourages iterative approach to design; controlled improvements, extensions and revisions of the artefacts move the system from one version to the next. 

The key to controlled evolution of a software system is to rely on rigorous testing and validation at all levels of the architecture: from testing individual units of code (Unit Testing), through testing interfaces between components (Integration Testing), to testing a completely integrated software system (System Testing). 

Recent trends in software development practices even encourage the tests of a functionality to be developed first (Test-Driven Development). It can be argued that such systems tend to have low-coupling, as their design needs to accommodate testing components in isolation, in arbitrary configurations and with control over the dependencies (Mocking, Dependency Injection).

In order to ensure the correct operation of a software system over time, all the levels of testing need to be automatic and fast, in order to encourage the developers to perform them locally, following every change (Automated Testing, Continuous Integration). Adhering to those principles reduces reliance on manual testing, prevents regression errors in existing functionality (Regression Testing) and improves release time.

The techniques and objectives described above apply well to deterministic systems; testing comes down to verifying that a set of inputs will produce a particular set of outputs, because the operation of such a system can be approximated with a deterministic-state machine. The difficulty occurs when dealing with non-deterministic systems; testing involves verifying that a particular set of inputs will produce a correct set of outputs with a statistically significant probability.

Trading Algorithms are an example of a non-deterministic system that operates in a highly non-deterministic environment. Trading Algorithms are designed to respond to the situation on the market and to have a certain expectation as to the effect their actions will have on the market.


\pagebreak

\section{Summary}
\begin{itemize}
\item Software is not like architecture: software systems are not built once and then maintained; they are usually developed over time, with new features constantly added on top of existing features and existing features reimplemented.
\item Automated Testing and Validation of a software system is very important, because it gives confidence in the system etc.
\item Traditional testing techniques help verify whether the system works as the designers think it should work; those apply very well to deterministic systems.
\item However, such techniques only go so far when applied to non-deterministic systems; such system's behaviour needs to be analysed using statistics to verify their behaviour.
\item To this end, 
\end{itemize}