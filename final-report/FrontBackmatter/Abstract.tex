\chapter*{Abstract}
At 3:40 PM on November 14\textsuperscript{th}, 2007, a buggy Credit Suisse proprietary algorithm (SmartWB) sent approximately 600,000 cancel/replace messages for non existent orders, triggerring 450,000 error messages from the NYSE and leading to a disruption of trading, with 131,000 messages frozen in queue that could not be processed and ultimately had to be deleted. Credit Suisse was fined \$150,000.~\cite{Nyse2009} On May 6, 2010, U.S. stock market has experienced a sudden price drop of 5\%, followed by a rapid recovery, in the mean time sending Accenture share price to a single cent and Sotheby’s to \$99,999.99, all in the course of about 30 minutes. Subsequent analyses of the incident concluded that the the incident was triggered by a large sell program that lead to "hot-potato" effect, where in the 14-second period more than 27,000 futures contracts were bought and sold, with the net effect of only around 200 contracts.~\cite{Kirilenko2011}

These prominent examples of software errors in Trading Algorithms highlight an urgent issue with the way these systems are tested. In order to prevent Trading Algorithms from exhibiting such behaviour, there is a need for a strategy of rigorous testing in a realistic environment. Current testing techniques (backtesting on historical data) fail to capture the dynamic nature of markets and hence do not provide an environment for effective flaw identification. In this thesis we demonstrate an Agent-Based Stock Market Simulator and evaluate its efficacy in automatic discovery of software errors in Trading Algorithms.
\vfill