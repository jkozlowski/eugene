\documentclass[ openright,titlepage,numbers=noenddot,headinclude,%twoside,
                footinclude=true,BCOR=5mm,paper=a4,fontsize=12pt,a4paper,english%
                ]{scrreprt}
\input{classicthesis-config}

\begin{document}


\begin{titlepage}
\begin{center}
\topskip0pt
\vspace*{\fill}
\rule{\linewidth}{0.5mm}
\Huge{Eugene} \\[0.5cm]
\LARGE{Agent-Based Market Simulator for use in Validation and System Testing of Trading Algorithms}
\rule{\linewidth}{0.5mm}\\[2cm]

\large{Jakub Koz\l{}owski \\
MEng Computer Science \\
Submission Date: 27th April 2012 \\
Internal Supervisor: Dr. Christopher D. Clack \\
External Supervisor: Ian Rose-Miller, UBS}
\vspace*{\fill}


\null
\vfill
\normalsize{This report is submitted as part requirement for the MEng Degree in Computer Science at UCL. It is substantially the result of my own work except where explicitly indicated in the text. The report may be freely copied and distributed provided the source is explicitly acknowledged.}
\end{center}
\end{titlepage}



\input{FrontBackmatter/Contents}
\chapter*{Abstract}
At 3:40 PM on November 14\textsuperscript{th}, 2007, a buggy Credit Suisse proprietary algorithm (SmartWB) sent approximately 600,000 cancel/replace messages for non existent orders, triggerring 450,000 error messages from the NYSE and leading to a disruption of trading, with 131,000 messages frozen in queue that could not be processed and ultimately had to be deleted. Credit Suisse was fined \$150,000.~\cite{Nyse2009} On May 6, 2010, U.S. stock market has experienced a sudden price drop of 5\%, followed by a rapid recovery, in the mean time sending Accenture share price to a single cent and Sotheby’s to \$99,999.99, all in the course of about 30 minutes. Subsequent analyses of the incident concluded that the the incident was triggered by a large sell program that lead to "hot-potato" effect, where in the 14-second period more than 27,000 futures contracts were bought and sold, with the net effect of only around 200 contracts.~\cite{Kirilenko2011}

These prominent examples of software errors in Trading Algorithms highlight an urgent issue with the way these systems are tested. In order to prevent Trading Algorithms from exhibiting such behaviour, there is a need for a strategy of rigorous testing in a realistic environment. Current testing techniques (backtesting on historical data) fail to capture the dynamic nature of markets and hence may fail to identify flaws in the Trading Algorithms. In this thesis we demonstrate an Agent-Based Stock Market Simulator and evaluate its efficacy in automatic discovery of software errors in Trading Algorithms.
\vfill

\pagestyle{scrheadings}
%\doublespacing

\chapter{Introduction}
\label{Chapters/Introduction}
\begin{itemize}
\item Outline the problem you are working on, why it is interesting and
what the challenges are.
\item List your aims and goals. An aim is something you intend to achieve (e.g., learn a new programming language and apply it in solving the problem), while a goal is something specific you expect to deliver (e.g., a working application with a particular set of features).
\item Give an overview of how you carried out the project (e.g., an iterative approach).
\item A brief overview of the rest of the chapters in the report (a guide to the reader of the overall structure of the report).
\end{itemize}
This chapter is relatively short (2-4 pages) and must leave the reader very clear on what the project is about and what your goals are.
\input{Chapters/Introduction/Motivation}
\section{Research Objective and Null Hypotheses}

\subsection{Research Objective}
We focus in this thesis on testing Trading Algorithms using Agent-Based Simulation and try to demonstrate that the results obtained from the simulations can be used to automatically detect simple programming errors using statistical analysis. 

In order to demonstrate the efficacy of this approach, we will study an implementation of a VWAP algorithm that will trade in a Simulated Stock Market that consists of Noise Traders. Given this VWAP algorithm implementation, a logical polarity error will be introduced into the implementation, which will cause the algorithm to always cross the spread, by checking the wrong side of the limit order book for a price when sending a limit order. We hypothesise that by performing statistical analysis on the distribution of errors between VWAPs achieved by both algorithms and those of the overall market, it will be possible to detect the error with high level of confidence.

\subsection{Null Hypothesis 1}
\begin{quote}
Given a correct implementation of a VWAP Algorithm, if a logical polarity error is introduced that causes the algorithm to always check the wrong price when sending a limit order, there will be no effect on the distribution of errors from the target VWAP.
\end{quote}

\subsection{Null Hypothesis 2}
\begin{quote}
Given 2 correct implementations of a VWAP Algorithm, one that divides the trading duration into 12 equal time intervals and the second that divides the trading duration into 24 equal time intervals, there will be no effect on the distribution of errors from the target VWAP.
\end{quote}





\section{Research methodology}

In order to design a stock market simulator, the first step is to study the relevant aspects of real stock markets, therefore we study the literature on market microstructure of electronic, limit order book stock markets.

Similarly, in order to realistically model the behaviour of a stock market as a whole, we turn to analyse existing Agent-Based Artificial Stock Markets. We specifically focus on the types of behaviours that are modelled in order to choose an appropriate implementation for the Noise Traders.

The results of the analysis of stock markets and Agent-Based Artificial Stock Markets enables us to design a flexible and modular framework for the simulator, that effectively mirrors the features of real stock markets. 

After designing and developing the simulator, the efficacy in discovering errors in Trading Algorithms will be evaluated by performing two experiments that correspond to the two null hypotheses. By introducing the simulator and demonstrating its usefulness in discovering errors in Trading Algorithms we aim to provide a methodological advancement that will be a useful complement to traditional ways of testing Trading Algorithms.

\section{Outline of the Report \label{Chapters/Introduction/Outline}}

\subsection*{Chapter \ref{Chapters/Background}} 
In Chapter \ref{Chapters/Background/Market-Architecture} we describe the model of the market mechanism we will use in the simulator. We describe the market architectures employed in the stock exchanges during continuous trading sessions, namely model of double auction, or limit order book, and how it enables market participants to interact asynchronously.

In Chapter \ref{Chapters/Background/Agent-Based-Modelling}



\chapter{Background}
\section{Market microstructure}
\label{Chapters/Background/Market-Microstructure}

In this section we describe the model of the market mechanism we will use in the simulator. We describe the market microstructure employed in major stock exchanges during continuous trading sessions, namely the model of a limit order book, and how it enables market participants to interact asynchronously (for a study of market microstructures in main stock exchanges, see~\cite{Comerton2004}). 


\subsection{Instruments}
Instruments are the different types of \textit{shares} that can be traded on a stock exchange, that represent ownership of a company. Every instrument is associated with a set of technicalities, such as the minimum tick size (minimum amount of money by which the price can change) or price variation controls (conditions for a market halt due to unexpected price volatility).

\subsection{Order Types}
Market participants indicate their willingness to trade in the form of trading instructions called \textit{orders}. We will consider only two types of orders: \textit{limit} and \textit{market} orders. A \textit{market order} specifies the instrument to trade, the quantity and the side of the trade (buy or sell). A \textit{limit order} additionally specifies a \textit{limit price}: the maximum (buy) or minimum (sell) price that the trader accepts for an order. 

Traders can also cancel existing orders that have not been executed. \citet{Lilo2004} estimate that on the London Stock Exchange (LSE) on-book market, up to 30\% of outstanding limit orders are cancelled before execution and order cancellations play an important role in the price formation process. 

\subsection{The limit order book}
The limit order book is the leading market mechanism used by main stock exchanges during continuous trading. A limit order book for a single instrument consists of limit orders, sorted by price and time of arrival and stored in two queues: one for bid (buy) orders and one for ask (sell) orders. At a specific time $t$, the order book can be described as~\cite{Gilles2006}: 
\begin{equation*}
\beta_n \leq \ldots \leq \beta_2 \leq \beta_1 < \alpha_1 \leq \alpha_2 \leq \ldots \alpha_m
\end{equation*}
where $\beta_i$ represent bid orders and $\alpha_j$ represent ask orders. The highest bid $\beta_1$ (or best bid) and lowest ask $\alpha_1$ (or best ask) define the spread $\alpha_1 - \beta_1$.

An incoming limit order can either trigger a trade or be stored in the book.    $\beta_1$ will be executed only if the book receives a market sell order, or a limit sell order with a limit price lower than or equal to $\beta_1$. In this case a trade will be executed at the price of $\beta_1$. Similarly, $\alpha_1$ will be executed only if the book receives a market buy order, or a limit buy order with a limit price higher than or equal to $\alpha_1$ in which case a trade will be executed at the price of $\alpha_1$.

\subsection{Transparency}
Market transparency is defined as the ability of market participants to observe information in the market. It can refer to two stages in the lifetime of an order: \textit{pre-trade} and \textit{post-trade} transparency. \textit{Pre-trade} transparency refers to the ability of other market participants to observe the limit orders entering the order book, whereas \textit{post-trade} transparency refers to observing trades after they have taken place.

The extent of \textit{pre-trade} transparency varies across different exchanges, but generally two levels of disclosure emerge: \textit{level 1} and \textit{level 2}. \textit{Level 1} usually refers to publishing best bid/ask quotes  with aggregate volumes, whereas \textit{level 2} discloses entire limit order book in real-time. In case of \textit{level 2} access, the broker IDs can either be disclosed or remain anonymous, depending on the stock exchange.

In terms of \textit{post-trade} transparency, \textit{immediate} reporting of trade executions is required, with the definition of \textit{immediate} varying between exchanges, however exceptions from this rule are possible. 















\section{Agent-Based Artificial Stock Markets}
\label{Chapters/Background/Agent-Based-Modelling}

In this section we summarise several artificial stock markets from the available literature. We provide a model of zero-intelligence agents that will be implemented by the Noise Traders, that is based on the work by~\citet[chap.~4]{Gilles2006}. Moreover, in order to provide a background material for implementing a VWAP Trading Algorithm for the experiments, we survey available sources, most prominently the work in~\cite{Coggins2006, Kakade2004}. 

\subsection{Agent-Based Modelling}
Agent-Based Modelling is a simulation technique concerned with designing societies of rule-based software agents that interact in particular ways, with a view of assessing the influence of individual (or groups of) agents on the system as a whole. This technique has been successfully applied in many business scenarios, including financial simulations. Among the benefits of agent-based modelling is emergent phenomena: behaviour resulting from complex interactions of many individual entities.

Agent-Based Models that attempt to explain economic processes are branded as Agent-Based Computational Economics. In recent years, studying stock markets using multi-agent based models has become a promising research area due to the fact that this methodology reflects the fundamental nature of a stock market, where the current situation is a result of a complex interaction of actions of many heterogenous investors that have various expectations and different levels of rationality.

We will now turn to review a number of existing Agent-Based Artificial Stock Markets.

\subsection{SantaFe ASM}
The Santa Fe Artificial Stock Market (SFI ASM)~\citep{Lebaron2002, Lebaron99} is a discrete time artificial stock market that consists of a central computational market and a number of intelligent agents. Agents make decisions by attempting to forecast the future returns on the stock using genetic programming and therefore decide between investing in risky stock or leaving their money in the bank, which pays a fixed interest rate.

The simulated market consists of $N$ agents, usually $50-100$, that interact with the market. The stock has the price $p(t)$ per share at time $t$, where $p(t)$ is set endogenously to clear the market (\textit{call market}). Similarly, the stock pays a dividend $d(t+1)$ at the end of time $t$, according to a stochastic process. The money left in the bank pays constant rate of return of $r$ per period. The information available to agents consists of the price, the dividend, total number of bids/asks at each past period, plus some additional variables. 

By using genetic programming, agents can explore a wide range of possible forecasting rules and they have the flexibility to use or disregard certain pieces of avaiable information. The market was able to generate the key stylised facts: weak forecastability, volatility persistence, and higher expected returns.~\cite{Lebaron99}. \marginpar{I should probably add some explanation of stylised facts somewhere above, because I am going to refer to those in all those ASMs} 

\subsection{Genoa ASM}
Similarly to SFI ASM, The Genoa Artificial Stock market (GASM)~\citep{Raberto2001} is an agent-based artificial financial market in which heterogeneous agents exchange cash and stock from an initial fixed endowment, therefore the total amount of cash and shares is constant throughout the duration of the simulation. The price formation process is set at the intersection of the demand and supply curves (\textit{call market}). At each time step, agents place random buy or sell orders, subject to available resources and clustering, for prices that depend on historical volatility.

The market was able to reproduce key stylised facts, i.e. fat tails and volatility clustering, however failed to reproduce others, e.g. the volatility exhibited an exponential decay, as opposed to power law decay.

\subsection{ABSTRACTE}
The ASMs surveyed thus far evolve in discrete time intervals, i.e. at each time period $t$, some fraction of traders submit orders to the market, the orders are cleared, a new market price is announced, the agents update their positions and a new time period $t+1$ starts. This style of simulation is well placed when dealing with call markets, however fails to realistically reflect the nature of continuous trading sessions, that evolves in a highly asynchronous manner thanks to limit order book microstructure.  

The importantance of asynchronous simulation, as opposed to discrete-time simulation was demonstrated by~\citet{Sorban2008}, who describes \textit{ABSTRACTE}: Agent-Based Simulation of Trading Roles in an Asynchronous Continuous Trading Environment. \textit{ABSTRACTE} was used to model a market with information asymmetry, where prices are set by a learning market maker; this model extends the work in~\citep{Das2006}, but evolves the simulation in continuous time. They show that moving to continuous, asynchronous time simulation renders different price dynamics and conclude that continuous nature of trading in real stock markets should be explicitly taken into account in agent-based models.

\subsection{Zero-intelligence model of price formation}

Another strong case for turning to asynchronous simulation, is presented by~\citet{Gilles2006}, who studied the primary role played by liquidity dynamics in price formation mechanism. Liquidity is initially modeled using a limit order book simulation populated with a zero-intelligence model of agents, that place random buy/sell and limit/market orders. The prices can either be sent within the spread or out of the spread and are drawn from uniform and power law distribution, respectively. The order quantities follow log-normal distribution. In order to model the well documented pattern of trading activity that is more dense at the start and end of the day~\cite{Clark1973}, the agents sleep for random times that are drawn from a stretched exponential distribution, that is obtained as a mixture of two exponential distributions with different means (the influence of sleep time on price dynamics is explored in~\cite{Scalas2004}). Similarly to~\cite{Raberto2001}, these zero-intelligence agents are endowed with a fixed amount of cash and shares.






   








\section{VWAP Trading}
Whenever a large institutional investor wants to take a position or liquidate a holding, the execution of the transaction is faced with price risk. Simply putting a large limit order onto the limit order book would give an incentive to other investors to change their prices and hence drive the cost of executing the order unnecessarily high. 

The solution is to split the parent order into smaller child orders in order to hide the intent of the investor and keep the market impact under control. Traditionally, such task would be delegated to human traders, however following the advent of algorithmic trading, it is usually a machine that executes the order. The quality of execution is measured by comparing against an appropriate benchmark.

Among the most popular benchmarks is \textit{VWAP} or \textit{Volume Weighted Average Price}, that is a measure of average price achieved in the market. When used to measure the quality of execution of an algorithm, the volume weighted prices achieved by the algorithm are compared to all the  other trades that occurred in the market during the period of the algorithm's activity. 

A \textit{VWAP Algorithm} attempts to buy or sell a fixed number of shares at a price that closely tracks the \textit{VWAP} of the market. Therefore, the problem of tracking the market \textit{VWAP} can be stated in terms of splitting the order into a series of smaller orders, whose size corresponds to the forecast intra-day volume pattern of a stock. 

\subsection{The Price-Volume Trading Model}
In \textit{price-volume trading model}~\cite{Kakade2004}, the \textit{intra-day} trading activity can be summarised by a discrete sequence of price and volume pairs $(p_t, v_t)$, for $t=1,\ldots,T$. Each pair represents the fact that a total volume of shares $v_t$ was traded at a price $p_t$. 

Assume that there is a trading algorithm $A$ that traded during this period.  Then, the market \textit{VWAP}, $VWAP_m$, for an intraday trading sequence $S_m = (p_1, v_1), \ldots, (p_T, v_T)$ that excludes the trades executed by the algorithm $A$, is then defined as follows:
\begin{equation}  
\label{Equation/Market-Vwap}
VWAP_m(S_m) = \frac{\left( \displaystyle\sum\limits_{t=0}^T (p_t, v_t) \right)}{\displaystyle\sum\limits_{t=0}^T v_tV}
\end{equation}

Similarly, the \textit{VWAP} the algorithm $A$, $VWAP_A$, for a sequence $S_A = (p_1, v_1), \ldots, (p_T, v_t)$ is defined as follows:
\begin{equation}  
\label{Equation/Algo-Vwap}
VWAP_A(S_A) = \frac{\left( \displaystyle\sum\limits_{t=0}^T (p_t, v_t) \right)}{\displaystyle\sum\limits_{t=0}^T v_t}
\end{equation}

Having defined what a \textit{VWAP} is, we will now turn to an implementation of a \textit{VWAP Algorithm}.

\subsection{VWAP Trading Algorithm under the price-volume trading model}
\citet{Coggins2006} define the following rule based approach to a \textit{VWAP} buy execution:
\begin{enumerate}
\item Divide the trading period into time slots, allocating a given percentage of trade volume to each time interval.
\item At each time slot, submit a limit order of the specified size at the best bid.
\item If within $x$ minutes, the best bid has gone up and our order has not executed, amend the order to the best bid.
\item If by the end of the time slot, the order has not fully completed, amend it to become a market order to force completion.
\end{enumerate}

This algorithm will be used in our experiments with \textit{VWAP algorithms}.






\chapter{Analysis and Design}
\label{Chapters/Analysis-and-Design}
In this Chapter we describe main elements of an Algorithmic Trading System and Eugene’s place in that architecture. We provide a summary of the main requirements and principles that informed the design of the simulator and the pseudo code for the algorithms implemented for the Noise Traders and VWAP Traders.
\section{Trading System Design} 

Various market participants submit orders to the stock exchange in order to trade. Different stock exchanges publish messages about the submitted/executed orders in a variety of formats. In order to deal with this complexity investment banks have developed internal market data buses that subscribe to feeds published by different stock exchanges in order to republish them in a standardised way to various internal systems, e.g. Trading Algorithms.

Similarly, in order to deal with complexity of submitting orders to different exchanges investment banks develop order management systems that register with various stock exchanges in order to provide a standardised way of submitting orders for various internal systems, e.g. Trading Algorithms, but also to allow Automatic Order Routing between different stock exchanges.

Historical data is stored in Tick Databases that subscribe to the market data bus in order to perform statistical analysis on the behaviour of different stocks. The result of the statistical analysis is a set of parameters which describe various aspects of stock behaviour. The parameters are published to various internal systems, e.g. Trading Algorithms.

Trading Algorithms are highly sophisticated and parameterised systems which accept parent orders and execute them using different strategies. Trading Algorithms register with the market data bus in order to react to the current situation on the market. They continuously compare the behaviour of stocks with their historical behaviour (using data and analysis from the tick database) in order to minimise market impact of the different strategies. Having described the architecture of a trading system inside an investment bank, we can now turn to analyse how to approach testing the trading algorithms that operate within this architecture.

\section{Simulating a Trading System}

Trading Algorithms respond to messages received from the Stock Exchange (via the Market Data Bus) and send orders to the Stock Exchange (via the Order Management System). They expect the orders to have some influence on the situation on the Stock Exchange, based on the analysis of past behaviour received from the Tick Database. In our implementation we are going to explicitly omit the issue of past behaviour analysis and the Tick Database, as it is beyond the scope of this report. 

Therefore, in order to simulate a Trading System for testing Trading Algorithms, two systems need to be put in place:
\begin{itemize}
\item Order Matching Engine which can accept orders on to a limit order book and match bid/ask orders.
\item Market behaviour simulator to generate a realistic order flow. \end{itemize}

Having stated the problem we are trying to solve and the required architecture, we will now present a formal set of requirements that the system needs to fulfil.

\FloatBarrier
\section{Requirements Analysis}
\label{Chapters/Analysis-and-Design/Requirements-Analysis}

Having analysed available literature on the design of asynchronous, limit order book markets in \Cref{Chapters/Background/Market-Microstructure}, as well as existing Agent-Based ASMs in \Cref{Chapters/Background/Agent-Based-Modelling}, we will now list the requirements that the trading simulator needs to satisfy. Requirements are prioritised according to the MoSCoW approach (M - Must, S - Should, C - Could and W - Would).

\subsection{Functional Requirements}
\begin{center}
\begin{longtable}[htbp]{c p{4.2in} c }

\multicolumn{1}{c}{\textbf{ID}}           &
\multicolumn{1}{c}{\textbf{Requirement}}  &
\multicolumn{1}{c}{\textbf{Priority}}     \\              
\toprule

\multicolumn{3}{c}{\textbf{Market Agent}}   	         \\
F01  & The Execution Engine shall match orders in the Order Book & M \\ 
F02  & The Market Agent shall accept limit/market orders and order cancellations. & M \\
F03  & The Market Agent shall provide \textit{level 2} market data access with anonymous OrderIDs to Trader Agents. & M \\
F04  & The Market Agent shall report order executions and status changes back to the original Trader Agent. & M \\ 
F05  & The Market Agent shall log all market events to a file.        & M \\
F06  & The Market Agent shall publish performance statistics at runtime. & C \\

\multicolumn{3}{c}{\textbf{Simulation Agent}}   \\
F07  & The Simulation Agent shall start  the Market Agent and Trader Agents.  & M \\
F08  & The Simulation Agent shall synchronise the start of a simulation. & M \\
F09  & The Simulation Agent shall stop a simulation after a specified amount of time. & M \\
F10  & The Simulation Agent shall send initial orders before starting the simulation. & M \\ 

\multicolumn{3}{c}{\textbf{Client API}}   \\
F11  & The Client API shall handle the low-level interaction with the Market Agent and Simulation Agent and provide callback methods for receiving messages. & M \\
         
\end{longtable}
\end{center}

\needspace{11\baselineskip}
\subsection{Non Functional Requirements}

\begin{table}[htbp]
\begin{center}
\begin{longtable}{c p{4.2in} c }

\multicolumn{1}{c}{\textbf{ID}}           &
\multicolumn{1}{c}{\textbf{Requirement}}  &
\multicolumn{1}{c}{\textbf{Priority}}     \\        
\toprule

N01  & The System shall maintain $\geq70\%$ unit test coverage.    & M \\
N02  & The System shall be integration tested.                     & M \\
N03  & The System shall be highly modular and decoupled.           & M \\
N04  & The System shall be very well javadoced.                    & M \\
N05  & The System shall be implemented on top of the JADE Framework.   & M \\
N06  & The System shall closely approximate message types and formats of industry standard messaging formats. & C \\
 
\end{longtable}
\end{center}
\end{table}


\FloatBarrier
\section{Use case analysis}


\newcounter{usecases}
\addtocounter{usecases}{1}

\newenvironment{usecase}[1]
  {
   \subsection{UC\arabic{usecases}: {#1}}%
   \stepcounter{usecases}%
   \begin{samepage}%
   \begin{list}{}%
   {\topsep=5em
   \renewcommand\makelabel[1]{\textbf{##1}\hfill}%
   \settowidth\labelwidth{\makelabel{Post-Conditions} + 0.5in}%
   \setlength\leftmargin{\labelwidth}
   \addtolength\leftmargin{\labelsep}}}
  {\end{list}%
   \end{samepage}}

% UC1   
\begin{usecase}{Start the simulation}
\item[Primary Actors] Simulation Agent
\item[Secondary Actors] Market Agent, Trader Agents
\item[Description] The Simulation Agent starts the simulation.
\item[Pre-conditions] 
\item[Flow of Events] 
\begin{enumerate}
\item The Simulation Agent starts the Market Agent.
\item The Market Agent starts and sends a Started message back to the Simulation Agent.
\item The Simulation Agent receives the Started message.
\item The Simulation Agent starts the Trader Agents.
\item The Trader Agents start and log on with the Market Agent.
\item The Trader Agents send a Logon Complete message back to the Simulation Agent.
\item The Simulation Agent receives the Logon Complete messages.
\item After receiving all the Logon Complete messages, the Simulation Agent sends initial limit orders to the Market Agent.
\item After receiving acknowledgements for the initial limit orders, the Simulation Agent sends Start messages to the Trader Agents.
\item The Trader Agents send Started messages back to the Simulation Agent.
\item The Simulation Agent receives Started messages.
\item After receiving all Started messages, the Simulation Agent sleeps until the end of the simulation.
\end{enumerate}
\item[Post-conditions] The Simulation Agent sleeps.
\item[Alternative Flows] None. 
\end{usecase}
  
% UC2
\begin{usecase}{Logon}
\item[Primary Actors] Trader Agent
\item[Secondary Actors] Market Agent
\item[Description] The Trader Agent sends a Logon message to the Market Agent.
\item[Pre-conditions] 
\item[Flow of Events] 
\begin{enumerate}
\item The Trader Agent sends a Logon message to the Market Agent
\item The Market Agent receives the message and stores the Trader Agent ID.
\item The Market Agent sends an acknowledgement message back to the Trader Agent.
\item The Trader Agent receives the acknowledgement message.
\end{enumerate}
\item[Post-conditions] The Trader Agent ID has been saved. \item[Alternative Flows] None.
\end{usecase}
  
% UC3 
\begin{usecase}{Send Limit Order}
\item[Primary Actors] Trader Agent
\item[Secondary Actors] Market Agent, Execution Engine
\item[Description] The Trader Agent sends a Limit Order message to the Market Agent.
\item[Pre-conditions] The Trader Agent has logged on with the Market Agent and received the Start message from the Simulation Agent.
\item[Flow of Events] 
\begin{enumerate}
\item The Trader Agent sends a New Limit Order message to the Market Agent
\item The Market Agent receives the message and forwards the limit order to the Execution Engine.
\item The Market Agent sends an acknowledgement message back to the Trader Agent.
\item The Trader Agent receives the acknowledgement message.
\end{enumerate}
\item[Post-conditions] The Execution Engine accepted the New Limit Order message.
\item[Alternative Flows] None
\end{usecase}

% UC4 
\begin{usecase}{Send Market Order}
\item[Primary Actors] Trader Agent
\item[Secondary Actors] Market Agent, Execution Engine
\item[Description] The Trader Agent sends a New Market Order message to the Market Agent.
\item[Pre-conditions] The Trader Agent has logged on with the Market Agent and received the Start message from the Simulation Agent.
\item[Flow of Events] 
\begin{enumerate}
\item The Trader Agent sends a New Market Order message to the Market Agent.
\item The Market Agent receives the message and forwards the market order to the Execution Engine.
\item The Market Agent sends an acknowledgement message back to the Trader Agent.
\item The Trader Agent receives the acknowledgement message.
\end{enumerate}
\item[Post-conditions] The Execution Engine accepted the New Market Order message.
\item[Alternative Flows] None
\end{usecase}

% UC5
\begin{usecase}{Cancel Order}
\item[Primary Actors] Trader Agent
\item[Secondary Actors] Market Agent, Execution Engine
\item[Description] The Trader Agent sends a Cancel Order message to the Market Agent.
\item[Pre-conditions] The Trader Agent has logged on with the Market Agent and received the Start message from the Simulation Agent.
\item[Flow of Events] 
\begin{enumerate}
\item The Trader Agent sends a Cancel Order message to the Market Agent.
\item The Market Agent receives the message, checks if the order exists and forwards the Cancel Order message to the Execution Engine.
\item The Market Agent sends an acknowledgement message back to the Trader Agent.
\item The Trader Agent receives the acknowledgement message.
\end{enumerate}
\item[Post-conditions] The Cancel Order message has been accepted by the Execution Engine.
\item[Alternative Flows] The order referred to in the Cancel Order message does not exist, therefore the Market Agent sends a Cancel Reject message back to the Trader Agent.
\end{usecase}

% UC6
\begin{usecase}{Process New Limit Order message}
\item[Primary Actors] Execution Engine
\item[Secondary Actors] Trader Agents
\item[Description] The Execution Engine processes a New Limit Order message.
\item[Pre-conditions] A New Limit Order message has been accepted by the Execution Engine.
\item[Flow of Events] 
\begin{enumerate}
\item The Execution Engine checks if there are matching limit orders on the opposite side of the book of the limit order to execute.
\item The Execution Engine executes all matching limit orders and sends Execution Report messages to the counter-parties and Order Executed messages to the rest of the Trader Agents.
\item If the limit order has not been filled, the Execution Engine puts the limit order into the limit order book and sends Add Order messages to all Trader Agents.
\end{enumerate}
\item[Post-conditions] The matching limit orders have been executed, appropriate messages have been sent and the remaining quantity has been put into the limit order book.
\item[Alternative Flows] None.
\end{usecase}

% UC7
\begin{usecase}{Process New Market Order message}
\item[Primary Actors] Execution Engine
\item[Secondary Actors] Trader Agents
\item[Description] The Execution Engine processes a New Market Order message.
\item[Pre-conditions] The Execution Engine accepted a New Market Order message.
\item[Flow of Events] 
\begin{enumerate}
\item The Execution Engine checks if there are matching limit orders on the opposite side of the book of the market order to execute.
\item The Execution Engine executes all matching limit orders and sends Execution Report messages to the counter-parties and Order Executed messages for the limit orders to the rest of the Trader Agents.
\item If the market order has not been filled, the Execution Engine sends a Order Canceled message for the remaining quantity back to the Trader Agent that sent the market order.
\end{enumerate}
\item[Post-conditions] The market order has been executed, matching limit orders have been executed, appropriate messages have been sent and the remaining quantity has been canceled.
\item[Alternative Flows] None.
\end{usecase}

% UC8   
\begin{usecase}{Process Cancel Order message}
\item[Primary Actors] Execution Engine
\item[Secondary Actors] Trader Agents
\item[Description] The Execution Engine processes a Cancel Order message.
\item[Pre-conditions] The Execution Engine accepted a Cancel Order message. 
\item[Flow of Events] 
\begin{enumerate}
\item The Execution Engine removes the limit order from the limit order book.
\item The Execution Engine sends an Order Canceled message for the remaining quantity back to the Trader agent that sent the limit order.
\item The Execution Engine sends a Delete Order message to all the Trader Agents.
\end{enumerate}
\item[Post-conditions] The limit order has been canceled.
\item[Alternative Flows] None.
\end{usecase}

% UC9   
\begin{usecase}{Stop the simulation}
\item[Primary Actors] Simulation Agent
\item[Secondary Actors] Trader Agents
\item[Description] The Simulation Agent stops the simulation.
\item[Pre-conditions] The simulation is running. 
\item[Flow of Events] 
\begin{enumerate}
\item The Simulation Agent wakes up.
\item The Simulation Agent sends a Stop message to the Trader Agents.
\item The Trader Agents stop and send Stopped messages back to the Simulation Agent.
\item The Simulation Agent receives the Stopped messages.
\item After receiving all the Stopped messages, the Simulation Agent stops.
\end{enumerate}
\item[Post-conditions] The simulation is stopped.
\item[Alternative Flows] 
\end{usecase}


\chapter{Implementation and Testing}
\label{Chapters/Implementation}
\section{Overall Principles}
\label{Chapters/Implementation/Overall-Principles}
In this section I will describe the overall programming and design principles that have informed the implementation of this system in order to later refer to them in the later sections that describe the particular modules of the system. The principles will be described in no particular order, as their scope usually cuts through different architectural levels.

\paragraph{Design by contract and assertions} 
\marginpar{Replace with an example from the codebase}
Input is checked for correctness and the contract for the method or constructor parameters is clearly documented. To that end, \texttt{com.google.common.base.Preconditions} class from \texttt{guava-libraries}\cite{guava} is used extensively. The following is an example of a method that checks the \texttt{String} parameter for \texttt{null reference} and asserts that its length is not zero:
\lstinputlisting[caption=Checking for \texttt{null reference} and length of the \textit{String} parameter]{Code/Implementation/Overall-Principles/checkNotNullExample.java}

This method would have corresponding tests that check whether the correctness of parameters is correctly verified:
\lstinputlisting[caption=The corresponding unit tests]{Code/Implementation/Overall-Principles/checkNotNullTestExample.java}

Similarly, post-conditions and invariants would be verified. Strict adherence to these principles allows for much safer code that is guaranteed to be used correctly.

\paragraph{Separation of concerns, programming to an interface and information hiding}
The code is separated into many modules, so that responsibility of each part of the code is very strictly defined. The APIs between different modules are defined in terms of interfaces and specific implementations are hidden behind factories and private packages. The following is an example from the \textit{market/book} module:
\lstinputlisting[caption=market/book/src/main/java/eugene/market/book/OrderBook.java]{Code/Implementation/Overall-Principles/OrderBook.java}
\lstinputlisting[caption=market/book/src/main/java/eugene/market/book/OrderBooks.java]{Code/Implementation/Overall-Principles/OrderBooks.java}

\paragraph{Immutable objects}
An object is considered immutable if its state cannot be changed after it is constructed. Reliance on immutable objects is considered a sound strategy for creating simple code. Immutable objects have an additional advantage of being naturally \textit{thread-safe}: since there is no state that can be changed, they can be safely shared by multiple threads, without interference. The following is an example from the \textit{market/book} module:
\lstinputlisting[caption=market/book/src/main/java/eugene/market/book/OrderBooks.java]{Code/Implementation/Overall-Principles/Order.java}

\section{Overall Architecture}
Before I describe the architecture of all the modules that make up \textit{eugene}, I will present the main technologies used in the implementation.

\subsection{Java Agent DEvelopment Framework}
\texttt{JADE} is a software framework for developing distributed, multi-agent software systems in Java. \texttt{JADE} can be deployed onto a set of containers (nodes) which together form a platform (cluster).

Each platform has a \texttt{Main Container} which holds two special agents:
\begin{itemize}
\item The \texttt{Agent Management System} (AMS) which can create and kill agents, kill containers and shut down the entire platform.
\item The \texttt{Directory Facilitator} (DF) which implements a \texttt{yellow pages} service.
\end{itemize} 

Each agent in \texttt{JADE} operates in a separate thread of control, therefore allowing independent, preemptive behaviour. \texttt{JADE}  has been designed to operate within an \texttt{OSGi} container, therefore making it easier to deploy.

\subsection{Open Services Gateway Initiative Framework}
\texttt{OSGi} is dynamic module and service system platform for Java. Application components (distributed as \texttt{bundles}) can be remotely managed without requiring a reboot of the entire container. A package management system enables developers to package public and private APIs within the same bundle, but only expose public APIs at runtime. The dynamic service system allows the components to discover the addition of new services and act accordingly.

\subsection{Simple Logging Facade for Java}
\texttt{SLF4J} is an abstraction for various logging frameworks that allows the user to plug in the desired implementation at deployment time. Combined with \texttt{logback} implementation, it is a very powerful and versatile logging framework. Among the most useful features is the ability to tag a log entry to indicate the type of the event being logged. Different \texttt{markers} can then be directed to separate files.

\section{jade-unit (jade-unit/)}
\label{Chapters/Implementation/Jade-Unit}
In order to fulfil requirements \textit{N01} and \textit{N02}, there needs to be a way to create and tear down an arbitrary number of \texttt{JADE} nodes, potentially within a single JVM. Such capability allows for reuse of established testing tools and frameworks. 

There exists a \texttt{Test Suite Framework} available through the official \texttt{JADE} website, however maintenance of two entirely different testing environments seemed like an unnecessary complication.

Unfortunately, \textit{JADE} contains a messaging module\footnote{To communicate with other nodes.} that opens a socket and renders creating multiple \texttt{JADE} nodes impossible, especially when tests are run in parallel. That is because \texttt{JADE} would attempt to open the same socket multiple times, and there was no simple way to randomise the choice of sockets to avoid clashes. Also, side effects in tests are considered bad testing practice.

Luckily, the \texttt{JADE} architecture is modular and hence the implementation of the messaging module can be replaced. The \texttt{jade-unit} extension implemented for this project provides a no-op messaging module that simply ignores any messages directed to it. This allows the integration tests to reuse the entire testing architecture for tests that need to run inside a \texttt{JADE} node. It is expected that the patch file with \texttt{jade-unit} extension will find its way to the \texttt{JADE} maintainers, in due time.

The \texttt{jade-unit} extension does not have an adverse effect on the quality of testing, because the messaging module replaced is used only for cross-node communication (i.e. to send messages to \texttt{Agent}s located in different nodes), and \texttt{JADE} is designed to handle this communication transparently (i.e. \texttt{Agent}s are not directly affected by their node location, or the node location of other \texttt{Agent}s).
\section{Order Book (market/book/)}
\label{Chapters/Implementation/Order-Book}
This module implements the data structure behind a limit order book and maintains the status of the limit orders. Due to this separation of concerns, the module is used both by the \texttt{Market~Agent} (\Cref{Chapters/Implementation/Market-Agent}) and \texttt{Client~API} (\Cref{Chapters/Implementation/Client-API}) modules (see \Cref{Chapters/Analysis-and-Design/Overall-Design} for an explanation why the \texttt{Client~API} needs to maintain a limit order book), thus contributing to requirement \textit{N03}.

Following the principles outlined in \Cref{Chapters/Implementation/Overall-Principles}, an interface is defined in \\ \texttt{eugene.market.book.OrderBook} that specifies the methods for inserting new orders, executing and cancelling existing orders, and inspecting the orders at the top of the book. There are two implementations: \texttt{eugene.market.book.impl.DefaultOrderBook} and a delegate \texttt{eugene.market.book.impl.ReadOnlyOrderBook}, both hidden in a private package behind the \texttt{eugene.market.book.OrderBooks} factory.  

Two classes represent an order: \texttt{Order} and \texttt{OrderStatus} (both located in the \texttt{eugene.market.book} package). The \texttt{Order} class maintains static information about an order (see \Cref{Chapters/Background/Order-Types}), whereas the \texttt{OrderStatus} class tracks the execution of an order. Both classes are immutable (see~\Cref{Chapters/Implementation/Overall-Principles}). The mutable state, i.e. the mapping from the current \texttt{OrderStatus} to \texttt{Order}, is maintained by the \texttt{OrderBook} implementations internally.


\section{Simulation Ontology (simulation/ontology/)}
\label{Chapters/Implementation/Simulation-Ontology}
This module defines message types necessary in order to implement use case \textit{UC1} and \textit{UC9}. All \texttt{Simulation Ontology} message types are listed in \Cref{Appendix/Simulation-Ontology}.
\section{Simulation Agent (simulation/agent/)}
\label{Chapters/Implementation/Simulation-Agent}
The \texttt{Simulation Agent} implements use cases \textit{UC1} and \textit{UC9}. It starts the \texttt{Market Agent}, awaits for messages that synchronise the start of \texttt{Trader Agents}, sends the initial orders and starts and stops the simulation. 

When starting the \texttt{Trader Agents}, \texttt{Simulation Agent} passes an implementation of \texttt{eugene.simulation.agent.Simulation} that defines methods for obtaining the address of the \texttt{Market Agent}, the \texttt{Simulation Agent} and the instrument definition (name, tick size and default price).

\section{Market Ontology (market/ontology/)}
\label{Chapters/Implementation/Market-Ontology}
This module defines message types necessary in order to implement use cases  \textit{UC2-UC8}. All message types and fields are listed in \Cref{Appendix/Market-Ontology}.

In order to fulfil requirement \textit{N06}, message names and fields are modelled after a subset of the \textit{Financial Information eXchange (FIX) protocol}~\cite{FIX5} for order management messages and \textit{BATS Multicast PITCH 2.X protocol}~\cite{BATSPITCH} for \textit{Level 2} market data feed. However, in the interest of adhering to the \textit{DRY} principle, both order management and market data feed messages in \texttt{Market Ontology} use the same format and, where possible, message fields from \textit{BATS Multicast PITCH 2.X protocol} are substituted with message fields from \textit{Financial Information eXchange (FIX) protocol}.

The actual implementation of \texttt{Market Ontology} is loosely based on the \textit{QuickFIX/J} message classes~\cite{QUICKFIXJ}, but simplified accordingly. Porting \textit{QuickFIX/J} messages to \textit{JADE} was attempted, however quickly abandoned as very time-consuming and without a guarantee of success. Implementing a simplified version of the \textit{QuickFIX/J} messages turned out to be much less involved. The task could, however, be attempted in the future, in which case the decision to keep the architecture similar presents itself as a clear advantage. 

In order to correctly deal with prices and explicitly manage the problem of rounding, \textit{QuickFIX/J} messages use the \textit{java.math.BigDecimal} class. However, \textit{JADE} does not natively support serialising \textit{java.math.BigDecimal} fields, therefore using those required implementing a new codec and thus a greater understanding of low-level details of \textit{JADE's} serialisation mechanism. Luckily, the problems were anticipated and appropriately planned for.
\section{Market Agent (market/agent/)}
\label{Chapters/Implementation/Market-Agent}
The \texttt{Market Agent} plays the role of the Stock Exchange with a central limit order book. The implementation is very clearly split between the messaging subsystem and the part that maintains the order book, executes orders etc., in order to maintain clear separation of concerns and improve ease of unit-testing. 

The messaging subsystem implementation is located in the \\\texttt{eugene.market.esma.impl.behaviours} package and consists of implementations of \texttt{jade.core.behaviours.Behaviour} class (see~\Cref{Chapters/Background/Tools-and-Frameworks}). The \texttt{OrderServer} accepts messages to create new orders, cancel existing orders and handle the logon, i.e. messages from the \texttt{Market Ontology} (\Cref{Chapters/Implementation/Market-Ontology}). Similarly, the \texttt{SimulationOntologyServer} deals with messages from the \texttt{Simulation Ontology} (\Cref{Chapters/Implementation/Simulation-Ontology}). Furthermore, the \texttt{MarketDataServer} deals with sending out trade confirmations to counter parties and market data events (by inspecting the \texttt{MarketDataEngine}, explained below). In order to satisfy requirement \textit{F05}, the \texttt{MarketDataServer} logs the events (using \textit{SLF4J}, see \Cref{Chapters/Background/Tools-and-Frameworks}). The different log files and the format of the logs are explained in~\Cref{Appendix/Logging}.

The second part of the \texttt{Market Agent} is located in the \\ \texttt{eugene.market.esma.impl.execution} package. The \texttt{ExecutionEngine} deals with managing the order book, accepting new orders, cancelling existing orders and matching orders. All the various parts of the \texttt{ExecutionEngine} are refactored to separate classes for ease of testing: \texttt{MatchingEngine} implements the matching algorithm (\Cref{Chapters/Background/Matching-Process}) and \texttt{InsertionValidator} checks whether a market order can be accepted or should be rejected (\Cref{Chapters/Background/Matching-Process}). Every change to the limit order book (inserting a new order, cancelling an existing order and executing an order) triggers an event that is recorded in the \texttt{MarketDataEngine} (\texttt{eugene.market.esma.impl.execution.data} package). The separation of the order characteristics (price, size, type in \texttt{Order}) from the current execution status (\texttt{OrderStatus}) in the \texttt{Order Book} module, greatly simplifies the implementation of the \texttt{MarketDataEngine}: the events can refer to a snapshot of a status of an order and this snapshot can be accessed at any time, as long as a reference to the appropriate \texttt{OrderStatus} object is maintained. 

The points of synchronisation between the messaging subsystem and the execution subsystem are maintained in two classes: the \texttt{Repository} (\texttt{eugene.market.esma.impl} package) and the \texttt{MarketDataEngine}.  The \texttt{Repository} maintains the mapping between current active orders and the owner traders; whenever an order is accepted by the \texttt{OrderServer} it is recorded in the \texttt{Repository}. Similarly, the \texttt{MarketDataServer} retrieves events from the \texttt{MarketDataEngine} and sends them out.

In order to achieve high incoming order rate to handle the required number of \texttt{Trader Agents} (see~\Cref{Chapters/Experiments-and-Results}), the \texttt{OrderServer} and the \texttt{SimulationOntologyServer} operate in a different thread than the \texttt{MarketDataServer}. Therefore, both the \texttt{Repository} and the \texttt{MarketDataEngine} need to be thread-safe (but not the rest of the classes, as they are not shared by different threads). Due to a low number of synchronisation points and clear separation between mutable and immutable state of the order book (\Cref{Chapters/Implementation/Order-Book}), going from single-threaded to multi-threaded design of the \texttt{Market Agent} was relatively straightforward.

In order to satisfy requirement \textit{F06}, various parts of the \texttt{Market Agent} publish performance statistics at runtime, using the \texttt{Metrics} library~\cite{Metrics}. Time to reply to messages and their rate, size of the incoming and outgoing queues, are all published to a \texttt{JMX} bean, that can be read using, e.g. \texttt{JConsole}~\cite{JConsole} that is bundled with \texttt{Java JDK}. Using those statistics we determined that the \texttt{Market Agent} can easily keep up with the required number of \texttt{Trader Agents} (see~\Cref{Chapters/Experiments-and-Results}).

\section{Client API}
\label{Chapters/Background/Client-API}

\chapter{Experiments and Results}
\label{Chapters/Testing-and-Validation}

\begin{itemize}
\item Describe your testing strategy (unit, functional, acceptance testing
and how they are carried out). How were test cases selected.
\item Examples of specific tests and how they were carried out (e.g., using mock objects to break dependencies). Focus on the interesting cases.
\item A summary of the test results and what coverage was achieved. The detailed test report(s) should appear in the appendix.
\end{itemize}

If your project requires substantial evaluation of data and results, or other forms of testing that are not code-based, then adapt this chapter to suit.
This chapter will typically be 2-4 pages in length but could be more de- pending on the depth of testing done.

\chapter{Conclusions and Further Work}
\label{Chapters/Conclusion-and-Further-Work}

\begin{itemize}
\item Summary.
\item Conclusions.
\item How to use it and take it further.
\end{itemize}


\begin{itemize}
\item A summary of what the project has achieved. Address each goal set
out in the introduction.
\item A critical evaluation of the results of the project (e.g., how well were the goals met, is the application fit for purpose, has good design and implementation practice been followed, was the right implementation technology chosen and so on).
\item Future work. How could the project be developed if you had another 6 months.
\item Wrap-up and final thoughts on your project.
\end{itemize}

This chapter is typically 2-4 pages long but could be longer if the project
work requires more extensive evaluation.

\appendix
\section{Simulation Ontology (simulation/ontology/)}
\label{Chapters/Implementation/Simulation-Ontology}
This module defines message types necessary in order to implement use case \textit{UC1} and \textit{UC9}. All \texttt{Simulation Ontology} message types are listed in \Cref{Appendix/Simulation-Ontology}.
\section{Market Ontology (market/ontology/)}
\label{Chapters/Implementation/Market-Ontology}
This module defines message types necessary in order to implement use cases  \textit{UC2-UC8}. All message types and fields are listed in \Cref{Appendix/Market-Ontology}.

In order to fulfil requirement \textit{N06}, message names and fields are modelled after a subset of the \textit{Financial Information eXchange (FIX) protocol}~\cite{FIX5} for order management messages and \textit{BATS Multicast PITCH 2.X protocol}~\cite{BATSPITCH} for \textit{Level 2} market data feed. However, in the interest of adhering to the \textit{DRY} principle, both order management and market data feed messages in \texttt{Market Ontology} use the same format and, where possible, message fields from \textit{BATS Multicast PITCH 2.X protocol} are substituted with message fields from \textit{Financial Information eXchange (FIX) protocol}.

The actual implementation of \texttt{Market Ontology} is loosely based on the \textit{QuickFIX/J} message classes~\cite{QUICKFIXJ}, but simplified accordingly. Porting \textit{QuickFIX/J} messages to \textit{JADE} was attempted, however quickly abandoned as very time-consuming and without a guarantee of success. Implementing a simplified version of the \textit{QuickFIX/J} messages turned out to be much less involved. The task could, however, be attempted in the future, in which case the decision to keep the architecture similar presents itself as a clear advantage. 

In order to correctly deal with prices and explicitly manage the problem of rounding, \textit{QuickFIX/J} messages use the \textit{java.math.BigDecimal} class. However, \textit{JADE} does not natively support serialising \textit{java.math.BigDecimal} fields, therefore using those required implementing a new codec and thus a greater understanding of low-level details of \textit{JADE's} serialisation mechanism. Luckily, the problems were anticipated and appropriately planned for.
\chapter{Appendix}

\begin{itemize}
\item User manual - how to use it.
\item System manual = how to compile, make changes etc.
\item If overflowing, drop page numbers on Appendixes.
\item Have code examples that are being referred form Implementation.
\item Experiment results. 
\end{itemize}


%\end{onehalfspace}

%\begin{singlespace}
%\begin{footnotesize}
%\begin{twocolumn}
\bibliographystyle{plainnat}
%\bibliographystyle{IEEEtran}
\bibliography{Bibliography}
%\end{twocolumn}
%\end{footnotesize}
%\end{singlespace}

\end{document}
