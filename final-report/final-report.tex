\documentclass[a4, 11pt, english]{report}
\usepackage[utf8]{inputenc}
\usepackage{listings}
\usepackage{todonotes}
\usepackage{fancyhdr}
\usepackage[utf8]{fontenc}
\usepackage{babel}
\usepackage{setspace}
\lstset{language=Java, numbers=left, frame=single}

%Set margins
%\usepackage[margin=2.75cm,paper=a4paper,tmargin=3cm,bmargin=3.4cm]{geometry}

%Graphics importer
\usepackage{graphicx}

%Link formatter
\usepackage{hyperref}


%% ----------------------------------------------------------------
%\usepackage{fontspec}
%%\setmainfont{Georgia} 
%\setmonofont[Scale=MatchLowercase]{Menlo}
%
%\setstretch{1.3}  % It is better to have smaller font and larger line spacing than the other way round
%
%\pagestyle{fancy}                       % Sets fancy header and footer
%\fancyfoot{}                            % Delete current footer settings
%
%%\renewcommand{\chaptermark}[1]{         % Lower Case Chapter marker style
%%  \markboth{\chaptername\ \thechapter.\ #1}}{}} %
%
%%\renewcommand{\sectionmark}[1]{         % Lower case Section marker style
%%  \markright{\thesection.\ #1}}         %
%
%\fancyhead[LE,RO]{\bfseries\thepage}    % Page number (boldface) in left on even
%% pages and right on odd pages
%\fancyhead[RE]{\bfseries\nouppercase{\leftmark}}      % Chapter in the right on even pages
%\fancyhead[LO]{\bfseries\nouppercase{\rightmark}}     % Section in the left on odd pages
%
%\usepackage{setspace}
%\usepackage{vmargin}
%% ----------------------------------------------------------------

%% ----------------------------------------------------------------
% Essential packages
\usepackage{amsmath, amssymb, amsthm}
\usepackage{graphicx,color}
\usepackage[left=1.5in, right=1in, top=1in, bottom=1in, includefoot, headheight=13.6pt]{geometry}
\usepackage[square, comma, numbers, sort&compress]{natbib}
\usepackage[T1]{fontenc}

% Optional customisation packages

% Set font
%\usepackage{mathpazo}
\usepackage{palatino}
%\usepackage{times}

\usepackage[hang, small, bf, margin=20pt, tableposition=top]{caption} 
\setlength{\abovecaptionskip}{0pt}

% Page layout
%\parindent 0pt
\parskip 0.5ex
\renewcommand{\baselinestretch}{1.33}
\numberwithin{equation}{section}
\renewcommand{\bibname}{References}
\renewcommand{\contentsname}{Contents}
%\pagenumbering{roman}
%\bibliographystyle{unsrtnat}

% Customising headers (optional) - see fancyhdr.pdf
\usepackage{fancyhdr}
\pagestyle{fancy}
\rhead{}
\lhead{\nouppercase{\textsc{\leftmark}}}
\renewcommand{\headrulewidth}{0pt}
\makeatletter
\renewcommand{\chaptermark}[1]{\markboth{\textsc{\@chapapp}\ \thechapter:\ #1}{}}
\makeatother

% Customising chapter headings (optional) - see sectsty.pdf
%\usepackage{sectsty}
%\chapterfont{\large\sc\centering}
%\chaptertitlefont{\centering}
%\subsubsectionfont{\centering}

%%-----------------------------------------------------------------------


%\title{\textbf{Eugene: Agent-Based Market Simulator for use in Validation and System Testing of Trading Algorithms}}
%\author{Jakub Kozłowski \and Internal Supervisor: Dr. Christopher D. Clack \and External Supervisor: Ian Rose-Miller, UBS}
%
%\date{\today}

\title{
\huge{\textbf{Eugene: Agent-Based Market Simulator \\for use in Validation and System Testing \\of Trading Algorithms}}\\[1.2cm]
\Large{Jakub Koz\l{}owski} \\
\Large{Internal Supervisor: Dr. Christopher D. Clack} \\
\Large{External Supervisor: Ian Rose-Miller, UBS} \\[1.2cm]
\Large{Department of Computer Science \\University College London} \\[1cm]
\Large{April 2011}
}
\author{} \date{}

\begin{document}

\maketitle

\tableofcontents

\begin{onehalfspace}

\chapter{Introduction}
\label{introduction}
\section{Motivation}

\todo{Add references to all the parenthesis}

The design of a software system gradually evolves over time, as the understanding of the problem domain improves and the business requirements change. The cost of change of a design of a software system \todo{Too many ofs} is not trivial, however it is comparably lower than in other Engineering Disciplines and can be controlled by adhering to principles of good design (Design Patterns). This propensity of software encourages iterative approach to design; controlled improvements, extensions and revisions of the artefacts move the system from one version to the next. 

The key to controlled evolution of a software system is to rely on rigorous testing and validation at all levels of the architecture: from testing individual units of code (Unit Testing), through testing interfaces between components (Integration Testing), to testing a completely integrated software system (System Testing). 

Recent trends in software development practices even encourage the tests of a functionality to be developed first (Test-Driven Development). It can be argued that such systems tend to have low-coupling, due to the necessity of being able to test components in isolation, in arbitrary configurations and with control over the dependencies (Mocking, Dependency Injection).

In order to ensure the correct operation of a software system over time, the different levels of testing need to be fast and automatic, in order to encourage the developers to perform them locally, following every change (Automated Testing, Continuous Integration). Adhering to those principles reduces reliance on manual testing, prevents regression errors in existing functionality (Regression Testing) and improves release time.



\pagebreak
In addition, testing is performed in order to achieve an array of objectives.
\begin{itemize}
\item Validating behaviour of the functionality under development (Test-Driven Development). 
\item Discovering regression errors in existing functionality, after changes have been made to the software system (Regression Testing).
\end{itemize}

The techniques and objectives described in the previous section apply fairly well to testing deterministic systems. Such systems can be approximated into deterministic state-machines and therefore testing comes down to verifying that a set of inputs will produce a particular set of outputs.

On the other hand, testing non-deterministic systems involves verifying that a particular set of inputs will produce a correct set of outputs with a statistically significant probability.

Trading Algorithms are an example of a non-deterministic system that operates in a non-deterministic environment. Trading Algorithms are designed to respond to the situation on the market and they also expect their actions to have an effect on it.

Difficult as it is, Automated Testing of Trading Algorithms is essential in order to ensure correct order execution and to reduce reliance on manual testing. Automated Testing will prevent the developers from deploying incorrect software and improve Time to Market.

\section{Summary}
\begin{itemize}
\item Software is not like architecture: software systems are not built once and then maintained; they are usually developed over time, with new features constantly added on top of existing features and existing features reimplemented.
\item Automated Testing and Validation of a software system is very important, because it gives confidence in the system etc.
\item Traditional testing techniques help verify whether the system works as the designers think it should work; those apply very well to deterministic systems.
\item However, such techniques only go so far when applied to non-deterministic systems; such system's behaviour needs to be analysed using statistics to verify their behaviour.
\item To this end, 
\end{itemize}
\section{Research Objective and Null Hypotheses}
\input{introduction/research-methodology}
\section{Outline of the Report}

\chapter{Background}
\label{background}

\chapter{Analysis and Design}
\label{analysis-and-design}
\section{Trading System Design}

%\begin{figure}[H]
%\centerline{\includegraphics[scale=0.6]{trading-system-design.png}}
%\caption{Trading System Architecture}
%\label{fig:trading-system-architecture}
%\end{figure}

\subsection{Stock Exchange}
Market Participants submit orders to the Stock Exchange in order to trade. Stock Exchanges publish messages about the submitted/executed orders.

\subsection{Market Data Bus}
Market Data Bus subscribes to messages published by different Stock Exchanges in order to publish them in a standardised way to various internal systems, e.g. Trading Algorithms.

\subsection{Order Management System}
Registers with various Stock Exchanges in order to provide a standardised way of submitting orders by various internal systems, e.g. Trading Algorithms, and also allow Automatic Order Routing between different Stock Exchanges.

\subsection{Tick Database}
Subscribes to the Market Data Bus in order to perform a statistical analysis on the behaviour of different stocks. The result of the statistical analysis is a set of parameters which describe various aspects of stock behaviour. The parameters are published to various internal systems, e.g. Trading Algorithms.

\subsection{Trading Algorithms}
Highly sophisticated and parameterised systems which accept orders and execute them using different strategies. Trading Algorithms register with the Market Data Bus in order to react to the current situation on the market. They continuously compare the behaviour of stocks with their historical behaviour (using data from the Tick Database) in order to quickly discover unexpected behaviour. 

\section{Testing Environment}

%\begin{figure}[H]
%\centerline{\includegraphics[scale=0.5]{architectural-description/automated-trading-testing-environment.png}}
%\caption{Automated Trading Algorithm Testing Environment}
%\label{fig:automated-trading-algorithm-testing-environment}
%\end{figure}

Trading Algorithms respond to messages received from the Stock Exchange (via the Market Data Bus) and send orders to the Stock Exchange (via the Order Management System). They expect the orders to have some influence on the situation on the Stock Exchange. 

Therefore, in order to setup a testing environment for Trading Algorithms, two systems need to be put in place in order to simulate the Trading System.

\subsection{Order Book Simulator}
Order Matching Engine which can accept orders in to Continuous Auction Trading and match Bid/Sell orders. The simulator should be programmable to handle any number of exchanges and stocks. Such an engine is already available for use and will not be the subject of this project.

\subsection{Market Behaviour Simulator}
Market Behaviour Simulator that can realistically simulate a broad range of market behaviours (e.g. rising and falling, flash crash, gaming). This project will propose a design of such a Market Simulator using an Agent-Based Approach.

\section{Eugene}

\subsection{Approach}
Agent-Based simulation is a powerful technique that has been successfully applied in many business scenarios, including financial simulations. Among the benefits of agent-based modelling is emergent phenomena: behaviour resulting from complex interactions of many individual entities. 

Therefore, whereas in a real stock market, the current situation is a result of a complex interaction of many market players, it is hypothesised that an interaction of several types of software agents will result in an emergence of realistic market behaviour.

\subsection{Technologies}
\subsubsection{Java Agent DEvelopment Framework}
\texttt{JADE} is a software framework for developing distributed, multi-agent software systems in Java. \texttt{JADE} can be deployed onto a set of containers (nodes) which together form a platform (cluster).

Each platform has a \texttt{Main Container} which holds two special agents:
\begin{itemize}
\item The \texttt{Agent Management System} (AMS) which can create and kill agents, kill containers and shut down the entire platform.
\item The \texttt{Directory Facilitator} (DF) which implements a \texttt{yellow pages} service.
\end{itemize} 

Each agent in \texttt{JADE} operates in a separate thread of control, therefore allowing independent, preemptive behaviour. \texttt{JADE}  has been designed to operate within an \texttt{OSGi} container, therefore making it easier to deploy.

\subsubsection{Open Services Gateway initiative framework}
\texttt{OSGi} is dynamic module and service system platform for Java. Application components (distributed as \texttt{bundles}) can be remotely managed without requiring a reboot of the entire container. A package management system enables developers to package public and private APIs within the same bundle, but only expose public APIs at runtime. The dynamic service system allows the components to discover the addition of new services and act accordingly.

\subsection{Simulation Structure}
\begin{itemize} 
\item Eugene will simulate one day of a continuous auction trading.
\item Eugene will aim to simulate a set of typical markets, according to parameters such as: Volume Curve, Price Volatility, Bid/Ask spread, average order size etc. (list not exhaustive).
\item Eugene will aim to simulate the following set of agents (list is not exhaustive):
\begin{itemize}
\item High Frequency Trading Agent: agent that moves in and out of short-term positions many times each day, to capture trading opportunities that may open up only for fractions of a second.
\item Gaming Agent: agent that is designed to try to trick Trading Algorithms in order to manipulate them for its own gain.
\item Random Agent: agent that will make random decisions to send mid, aggressive or passive orders.
\item Technical Analysis Agent: agent that will make decisions based on past behaviour of the market.
\item etc.
\end{itemize}
\end{itemize}

\subsection{Aims}
\begin{itemize}
\item Analyse existing Agent-Based approaches to Market Simulation to determine a set of possible Agent Types.
\item Implement the Agents.
\item Discover a specific configuration of Agents whose behaviour will simulate each target market, in regards to a set of pre-determined market characteristics (Volume Curve, Price Volatility etc.).
\item Analyse efficacy of the Agent-Based Market Simulation approach for use in System Testing of Trading Algorithms.
\end{itemize}

\subsection{Success Factors}
\begin{itemize}
\item Develop a realistic simulation for at least one target market.
\end{itemize}


\chapter{Implementation}
\label{implementation}

\section{Things to draw attention to}
\begin{itemize}
\item The principle of defensive programming.
\item Separation of concerns with extensive use of programming to an interface.
\item Information hiding with use of private packages (OSGi) that contain implementations;
\item Hiding implementations behind factories.
\item Modularised design.
\item Extensive unit and integration testing.
\item jade-unit.
\item mocking.
\item Careful consideration of threading issues and hence multithreaded design.

\end{itemize}


\chapter{Testing and Validation}
\label{testing}

\chapter{Conclusion and Further Work}
\label{conclusion}

\end{onehalfspace}

\bibliographystyle{plain}
\bibliography{references}

\end{document}