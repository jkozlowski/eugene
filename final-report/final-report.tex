\documentclass[ openright,titlepage,numbers=noenddot,headinclude,%twoside,
                footinclude=true,BCOR=5mm,paper=a4,fontsize=12pt,a4paper,english%
                ]{scrreprt}
\input{classicthesis-config}

\begin{document}


\begin{titlepage}
\begin{center}
\topskip0pt
\vspace*{\fill}
\rule{\linewidth}{0.5mm}
\Huge{Eugene} \\[0.5cm]
\LARGE{Agent-Based Market Simulator for use in Validation and System Testing of Trading Algorithms}
\rule{\linewidth}{0.5mm}\\[2cm]

\large{Jakub Koz\l{}owski \\
MEng Computer Science \\
Submission Date: 27th April 2012 \\
Internal Supervisor: Dr. Christopher D. Clack \\
External Supervisor: Ian Rose-Miller, UBS}
\vspace*{\fill}


\null
\vfill
\normalsize{This report is submitted as part requirement for the MEng Degree in Computer Science at UCL. It is substantially the result of my own work except where explicitly indicated in the text. The report may be freely copied and distributed provided the source is explicitly acknowledged.}
\end{center}
\end{titlepage}



\input{FrontBackmatter/Contents}
\chapter*{Abstract}
At 3:40 PM on November 14\textsuperscript{th}, 2007, a buggy Credit Suisse proprietary algorithm (SmartWB) sent approximately 600,000 cancel/replace messages for non existent orders, triggerring 450,000 error messages from the NYSE and leading to a disruption of trading, with 131,000 messages frozen in queue that could not be processed and ultimately had to be deleted. Credit Suisse was fined \$150,000.~\cite{Nyse2009} On May 6, 2010, U.S. stock market has experienced a sudden price drop of 5\%, followed by a rapid recovery, in the mean time sending Accenture share price to a single cent and Sotheby’s to \$99,999.99, all in the course of about 30 minutes. Subsequent analyses of the incident concluded that the the incident was triggered by a large sell program that lead to "hot-potato" effect, where in the 14-second period more than 27,000 futures contracts were bought and sold, with the net effect of only around 200 contracts.~\cite{Kirilenko2011}

These prominent examples of software errors in Trading Algorithms highlight an urgent issue with the way these systems are tested. In order to prevent Trading Algorithms from exhibiting such behaviour, there is a need for a strategy of rigorous testing in a realistic environment. Current testing techniques (backtesting on historical data) fail to capture the dynamic nature of markets and hence may fail to identify flaws in the Trading Algorithms. In this thesis we demonstrate an Agent-Based Stock Market Simulator and evaluate its efficacy in automatic discovery of software errors in Trading Algorithms.
\vfill

\pagestyle{scrheadings}
%\doublespacing

\chapter{Introduction}
\label{Chapters/Introduction}
\begin{itemize}
\item Outline the problem you are working on, why it is interesting and
what the challenges are.
\item List your aims and goals. An aim is something you intend to achieve (e.g., learn a new programming language and apply it in solving the problem), while a goal is something specific you expect to deliver (e.g., a working application with a particular set of features).
\item Give an overview of how you carried out the project (e.g., an iterative approach).
\item A brief overview of the rest of the chapters in the report (a guide to the reader of the overall structure of the report).
\end{itemize}
This chapter is relatively short (2-4 pages) and must leave the reader very clear on what the project is about and what your goals are.
\input{Chapters/Introduction/Motivation}
\section{Research Objective and Null Hypotheses}

\subsection{Research Objective}
We focus in this thesis on testing Trading Algorithms using Agent-Based Simulation and try to demonstrate that the results obtained from the simulations can be used to automatically detect simple programming errors using statistical analysis. 

In order to demonstrate the efficacy of this approach, we will study an implementation of a VWAP algorithm that will trade in a Simulated Stock Market that consists of Noise Traders. Given this VWAP algorithm implementation, a logical polarity error will be introduced into the implementation, which will cause the algorithm to always cross the spread, by checking the wrong side of the limit order book for a price when sending a limit order. We hypothesise that by performing statistical analysis on the distribution of errors between VWAPs achieved by both algorithms and those of the overall market, it will be possible to detect the error with high level of confidence.

\subsection{Null Hypothesis 1}
\begin{quote}
Given a correct implementation of a VWAP Algorithm, if a logical polarity error is introduced that causes the algorithm to always check the wrong price when sending a limit order, there will be no effect on the distribution of errors from the target VWAP.
\end{quote}

\subsection{Null Hypothesis 2}
\begin{quote}
Given 2 correct implementations of a VWAP Algorithm, one that divides the trading duration into 12 equal time intervals and the second that divides the trading duration into 24 equal time intervals, there will be no effect on the distribution of errors from the target VWAP.
\end{quote}





\section{Research methodology}

In order to design a stock market simulator, the first step is to study the relevant aspects of real stock markets, therefore we study the literature on market microstructure of electronic, limit order book stock markets.

Similarly, in order to realistically model the behaviour of a stock market as a whole, we turn to analyse existing Agent-Based Artificial Stock Markets. We specifically focus on the types of behaviours that are modelled in order to choose an appropriate implementation for the Noise Traders.

The results of the analysis of stock markets and Agent-Based Artificial Stock Markets enables us to design a flexible and modular framework for the simulator, that effectively mirrors the features of real stock markets. 

After designing and developing the simulator, the efficacy in discovering errors in Trading Algorithms will be evaluated by performing two experiments that correspond to the two null hypotheses. By introducing the simulator and demonstrating its usefulness in discovering errors in Trading Algorithms we aim to provide a methodological advancement that will be a useful complement to traditional ways of testing Trading Algorithms.

\section{Outline of the Report \label{Chapters/Introduction/Outline}}

\subsection*{Chapter \ref{Chapters/Background}} 
In Chapter \ref{Chapters/Background/Market-Architecture} we describe the model of the market mechanism we will use in the simulator. We describe the market architectures employed in the stock exchanges during continuous trading sessions, namely model of double auction, or limit order book, and how it enables market participants to interact asynchronously.

In Chapter \ref{Chapters/Background/Agent-Based-Modelling}



\chapter{Background}
\label{Chapters/Background}
\begin{itemize}
\item Outline and reference the sources of information you are drawing on
(papers, books, websites, etc.). State how each relates to your work.
\item If relevant, a survey of similar programs or applications to yours, and how yours is differentiated.
\item Outline the software, tools, library code, frameworks and similar that you are using.
\end{itemize}
You should not include well known things (e.g., HTML or Java) or try to give tutorials on how to use a tool or code library (use references to books and websites for that information). Everything you include should be directly relevant to your work and the relationship made clear. This chapter is likely to be fairly substantial, perhaps 8-10 pages.
\section{Market microstructure}
\label{Chapters/Background/Market-Microstructure}

In this section we describe the model of the market mechanism we will use in the simulator. We describe the market microstructure employed in major stock exchanges during continuous trading sessions, namely the model of a limit order book, and how it enables market participants to interact asynchronously (for a study of market microstructures in main stock exchanges, see~\cite{Comerton2004}). 


\subsection{Instruments}
Instruments are the different types of \textit{shares} that can be traded on a stock exchange, that represent ownership of a company. Every instrument is associated with a set of technicalities, such as the minimum tick size (minimum amount of money by which the price can change) or price variation controls (conditions for a market halt due to unexpected price volatility).

\subsection{Order Types}
Market participants indicate their willingness to trade in the form of trading instructions called \textit{orders}. We will consider only two types of orders: \textit{limit} and \textit{market} orders. A \textit{market order} specifies the instrument to trade, the quantity and the side of the trade (buy or sell). A \textit{limit order} additionally specifies a \textit{limit price}: the maximum (buy) or minimum (sell) price that the trader accepts for an order. 

Traders can also cancel existing orders that have not been executed. \citet{Lilo2004} estimate that on the London Stock Exchange (LSE) on-book market, up to 30\% of outstanding limit orders are cancelled before execution and order cancellations play an important role in the price formation process. 

\subsection{The limit order book}
The limit order book is the leading market mechanism used by main stock exchanges during continuous trading. A limit order book for a single instrument consists of limit orders, sorted by price and time of arrival and stored in two queues: one for bid (buy) orders and one for ask (sell) orders. At a specific time $t$, the order book can be described as~\cite{Gilles2006}: 
\begin{equation*}
\beta_n \leq \ldots \leq \beta_2 \leq \beta_1 < \alpha_1 \leq \alpha_2 \leq \ldots \alpha_m
\end{equation*}
where $\beta_i$ represent bid orders and $\alpha_j$ represent ask orders. The highest bid $\beta_1$ (or best bid) and lowest ask $\alpha_1$ (or best ask) define the spread $\alpha_1 - \beta_1$.

An incoming limit order can either trigger a trade or be stored in the book.    $\beta_1$ will be executed only if the book receives a market sell order, or a limit sell order with a limit price lower than or equal to $\beta_1$. In this case a trade will be executed at the price of $\beta_1$. Similarly, $\alpha_1$ will be executed only if the book receives a market buy order, or a limit buy order with a limit price higher than or equal to $\alpha_1$ in which case a trade will be executed at the price of $\alpha_1$.

\subsection{Transparency}
Market transparency is defined as the ability of market participants to observe information in the market. It can refer to two stages in the lifetime of an order: \textit{pre-trade} and \textit{post-trade} transparency. \textit{Pre-trade} transparency refers to the ability of other market participants to observe the limit orders entering the order book, whereas \textit{post-trade} transparency refers to observing trades after they have taken place.

The extent of \textit{pre-trade} transparency varies across different exchanges, but generally two levels of disclosure emerge: \textit{level 1} and \textit{level 2}. \textit{Level 1} usually refers to publishing best bid/ask quotes  with aggregate volumes, whereas \textit{level 2} discloses entire limit order book in real-time. In case of \textit{level 2} access, the broker IDs can either be disclosed or remain anonymous, depending on the stock exchange.

In terms of \textit{post-trade} transparency, \textit{immediate} reporting of trade executions is required, with the definition of \textit{immediate} varying between exchanges, however exceptions from this rule are possible. 















\section{Agent-Based Artificial Stock Markets}
\label{Chapters/Background/Agent-Based-Modelling}

In this section we summarise several artificial stock markets from the available literature. We provide a model of zero-intelligence agents that will be implemented by the Noise Traders, that is based on the work by~\citet[chap.~4]{Gilles2006}. Moreover, in order to provide a background material for implementing a VWAP Trading Algorithm for the experiments, we survey available sources, most prominently the work in~\cite{Coggins2006, Kakade2004}. 

\subsection{Agent-Based Modelling}
Agent-Based Modelling is a simulation technique concerned with designing societies of rule-based software agents that interact in particular ways, with a view of assessing the influence of individual (or groups of) agents on the system as a whole. This technique has been successfully applied in many business scenarios, including financial simulations. Among the benefits of agent-based modelling is emergent phenomena: behaviour resulting from complex interactions of many individual entities.

Agent-Based Models that attempt to explain economic processes are branded as Agent-Based Computational Economics. In recent years, studying stock markets using multi-agent based models has become a promising research area due to the fact that this methodology reflects the fundamental nature of a stock market, where the current situation is a result of a complex interaction of actions of many heterogenous investors that have various expectations and different levels of rationality.

We will now turn to review a number of existing Agent-Based Artificial Stock Markets.

\subsection{SantaFe ASM}
The Santa Fe Artificial Stock Market (SFI ASM)~\citep{Lebaron2002, Lebaron99} is a discrete time artificial stock market that consists of a central computational market and a number of intelligent agents. Agents make decisions by attempting to forecast the future returns on the stock using genetic programming and therefore decide between investing in risky stock or leaving their money in the bank, which pays a fixed interest rate.

The simulated market consists of $N$ agents, usually $50-100$, that interact with the market. The stock has the price $p(t)$ per share at time $t$, where $p(t)$ is set endogenously to clear the market (\textit{call market}). Similarly, the stock pays a dividend $d(t+1)$ at the end of time $t$, according to a stochastic process. The money left in the bank pays constant rate of return of $r$ per period. The information available to agents consists of the price, the dividend, total number of bids/asks at each past period, plus some additional variables. 

By using genetic programming, agents can explore a wide range of possible forecasting rules and they have the flexibility to use or disregard certain pieces of avaiable information. The market was able to generate the key stylised facts: weak forecastability, volatility persistence, and higher expected returns.~\cite{Lebaron99}. \marginpar{I should probably add some explanation of stylised facts somewhere above, because I am going to refer to those in all those ASMs} 

\subsection{Genoa ASM}
Similarly to SFI ASM, The Genoa Artificial Stock market (GASM)~\citep{Raberto2001} is an agent-based artificial financial market in which heterogeneous agents exchange cash and stock from an initial fixed endowment, therefore the total amount of cash and shares is constant throughout the duration of the simulation. The price formation process is set at the intersection of the demand and supply curves (\textit{call market}). At each time step, agents place random buy or sell orders, subject to available resources and clustering, for prices that depend on historical volatility.

The market was able to reproduce key stylised facts, i.e. fat tails and volatility clustering, however failed to reproduce others, e.g. the volatility exhibited an exponential decay, as opposed to power law decay.

\subsection{ABSTRACTE}
The ASMs surveyed thus far evolve in discrete time intervals, i.e. at each time period $t$, some fraction of traders submit orders to the market, the orders are cleared, a new market price is announced, the agents update their positions and a new time period $t+1$ starts. This style of simulation is well placed when dealing with call markets, however fails to realistically reflect the nature of continuous trading sessions, that evolves in a highly asynchronous manner thanks to limit order book microstructure.  

The importantance of asynchronous simulation, as opposed to discrete-time simulation was demonstrated by~\citet{Sorban2008}, who describes \textit{ABSTRACTE}: Agent-Based Simulation of Trading Roles in an Asynchronous Continuous Trading Environment. \textit{ABSTRACTE} was used to model a market with information asymmetry, where prices are set by a learning market maker; this model extends the work in~\citep{Das2006}, but evolves the simulation in continuous time. They show that moving to continuous, asynchronous time simulation renders different price dynamics and conclude that continuous nature of trading in real stock markets should be explicitly taken into account in agent-based models.

\subsection{Zero-intelligence model of price formation}

Another strong case for turning to asynchronous simulation, is presented by~\citet{Gilles2006}, who studied the primary role played by liquidity dynamics in price formation mechanism. Liquidity is initially modeled using a limit order book simulation populated with a zero-intelligence model of agents, that place random buy/sell and limit/market orders. The prices can either be sent within the spread or out of the spread and are drawn from uniform and power law distribution, respectively. The order quantities follow log-normal distribution. In order to model the well documented pattern of trading activity that is more dense at the start and end of the day~\cite{Clark1973}, the agents sleep for random times that are drawn from a stretched exponential distribution, that is obtained as a mixture of two exponential distributions with different means (the influence of sleep time on price dynamics is explored in~\cite{Scalas2004}). Similarly to~\cite{Raberto2001}, these zero-intelligence agents are endowed with a fixed amount of cash and shares.






   








\section{VWAP Trading}
Whenever a large institutional investor wants to take a position or liquidate a holding, the execution of the transaction is faced with price risk. Simply putting a large limit order onto the limit order book would give an incentive to other investors to change their prices and hence drive the cost of executing the order unnecessarily high. 

The solution is to split the parent order into smaller child orders in order to hide the intent of the investor and keep the market impact under control. Traditionally, such task would be delegated to human traders, however following the advent of algorithmic trading, it is usually a machine that executes the order. The quality of execution is measured by comparing against an appropriate benchmark.

Among the most popular benchmarks is \textit{VWAP} or \textit{Volume Weighted Average Price}, that is a measure of average price achieved in the market. When used to measure the quality of execution of an algorithm, the volume weighted prices achieved by the algorithm are compared to all the  other trades that occurred in the market during the period of the algorithm's activity. 

A \textit{VWAP Algorithm} attempts to buy or sell a fixed number of shares at a price that closely tracks the \textit{VWAP} of the market. Therefore, the problem of tracking the market \textit{VWAP} can be stated in terms of splitting the order into a series of smaller orders, whose size corresponds to the forecast intra-day volume pattern of a stock. 

\subsection{The Price-Volume Trading Model}
In \textit{price-volume trading model}~\cite{Kakade2004}, the \textit{intra-day} trading activity can be summarised by a discrete sequence of price and volume pairs $(p_t, v_t)$, for $t=1,\ldots,T$. Each pair represents the fact that a total volume of shares $v_t$ was traded at a price $p_t$. 

Assume that there is a trading algorithm $A$ that traded during this period.  Then, the market \textit{VWAP}, $VWAP_m$, for an intraday trading sequence $S_m = (p_1, v_1), \ldots, (p_T, v_T)$ that excludes the trades executed by the algorithm $A$, is then defined as follows:
\begin{equation}  
\label{Equation/Market-Vwap}
VWAP_m(S_m) = \frac{\left( \displaystyle\sum\limits_{t=0}^T (p_t, v_t) \right)}{\displaystyle\sum\limits_{t=0}^T v_tV}
\end{equation}

Similarly, the \textit{VWAP} the algorithm $A$, $VWAP_A$, for a sequence $S_A = (p_1, v_1), \ldots, (p_T, v_t)$ is defined as follows:
\begin{equation}  
\label{Equation/Algo-Vwap}
VWAP_A(S_A) = \frac{\left( \displaystyle\sum\limits_{t=0}^T (p_t, v_t) \right)}{\displaystyle\sum\limits_{t=0}^T v_t}
\end{equation}

Having defined what a \textit{VWAP} is, we will now turn to an implementation of a \textit{VWAP Algorithm}.

\subsection{VWAP Trading Algorithm under the price-volume trading model}
\citet{Coggins2006} define the following rule based approach to a \textit{VWAP} buy execution:
\begin{enumerate}
\item Divide the trading period into time slots, allocating a given percentage of trade volume to each time interval.
\item At each time slot, submit a limit order of the specified size at the best bid.
\item If within $x$ minutes, the best bid has gone up and our order has not executed, amend the order to the best bid.
\item If by the end of the time slot, the order has not fully completed, amend it to become a market order to force completion.
\end{enumerate}

This algorithm will be used in our experiments with \textit{VWAP algorithms}.






\chapter{Analysis and Design}
\label{Chapters/Analysis-and-Design}
In this Chapter we describe main elements of an Algorithmic Trading System and Eugene’s place in that architecture. We provide a summary of the main requirements and principles that informed the design of the simulator and the pseudo code for the algorithms implemented for the Noise Traders and VWAP Traders.
\section{Trading System Design} 

Various market participants submit orders to the stock exchange in order to trade. Different stock exchanges publish messages about the submitted/executed orders in a variety of formats. In order to deal with this complexity investment banks have developed internal market data buses that subscribe to feeds published by different stock exchanges in order to republish them in a standardised way to various internal systems, e.g. Trading Algorithms.

Similarly, in order to deal with complexity of submitting orders to different exchanges investment banks develop order management systems that register with various stock exchanges in order to provide a standardised way of submitting orders for various internal systems, e.g. Trading Algorithms, but also to allow Automatic Order Routing between different stock exchanges.

Historical data is stored in Tick Databases that subscribe to the market data bus in order to perform statistical analysis on the behaviour of different stocks. The result of the statistical analysis is a set of parameters which describe various aspects of stock behaviour. The parameters are published to various internal systems, e.g. Trading Algorithms.

Trading Algorithms are highly sophisticated and parameterised systems which accept parent orders and execute them using different strategies. Trading Algorithms register with the market data bus in order to react to the current situation on the market. They continuously compare the behaviour of stocks with their historical behaviour (using data and analysis from the tick database) in order to minimise market impact of the different strategies. Having described the architecture of a trading system inside an investment bank, we can now turn to analyse how to approach testing the trading algorithms that operate within this architecture.

\section{Simulating a Trading System}

Trading Algorithms respond to messages received from the Stock Exchange (via the Market Data Bus) and send orders to the Stock Exchange (via the Order Management System). They expect the orders to have some influence on the situation on the Stock Exchange, based on the analysis of past behaviour received from the Tick Database. In our implementation we are going to explicitly omit the issue of past behaviour analysis and the Tick Database, as it is beyond the scope of this report. 

Therefore, in order to simulate a Trading System for testing Trading Algorithms, two systems need to be put in place:
\begin{itemize}
\item Order Matching Engine which can accept orders on to a limit order book and match bid/ask orders.
\item Market behaviour simulator to generate a realistic order flow. \end{itemize}

Having stated the problem we are trying to solve and the required architecture, we will now present a formal set of requirements that the system needs to fulfil.

\FloatBarrier
\section{Requirements Analysis}
\label{Chapters/Analysis-and-Design/Requirements-Analysis}

Having analysed available literature on the design of asynchronous, limit order book markets in \Cref{Chapters/Background/Market-Microstructure}, as well as existing Agent-Based ASMs in \Cref{Chapters/Background/Agent-Based-Modelling}, we will now list the requirements that the trading simulator needs to satisfy. Requirements are prioritised according to the MoSCoW approach (M - Must, S - Should, C - Could and W - Would).

\subsection{Functional Requirements}
\begin{center}
\begin{longtable}[htbp]{c p{4.2in} c }

\multicolumn{1}{c}{\textbf{ID}}           &
\multicolumn{1}{c}{\textbf{Requirement}}  &
\multicolumn{1}{c}{\textbf{Priority}}     \\              
\toprule

\multicolumn{3}{c}{\textbf{Market Agent}}   	         \\
F01  & The Execution Engine shall match orders in the Order Book & M \\ 
F02  & The Market Agent shall accept limit/market orders and order cancellations. & M \\
F03  & The Market Agent shall provide \textit{level 2} market data access with anonymous OrderIDs to Trader Agents. & M \\
F04  & The Market Agent shall report order executions and status changes back to the original Trader Agent. & M \\ 
F05  & The Market Agent shall log all market events to a file.        & M \\
F06  & The Market Agent shall publish performance statistics at runtime. & C \\

\multicolumn{3}{c}{\textbf{Simulation Agent}}   \\
F07  & The Simulation Agent shall start  the Market Agent and Trader Agents.  & M \\
F08  & The Simulation Agent shall synchronise the start of a simulation. & M \\
F09  & The Simulation Agent shall stop a simulation after a specified amount of time. & M \\
F10  & The Simulation Agent shall send initial orders before starting the simulation. & M \\ 

\multicolumn{3}{c}{\textbf{Client API}}   \\
F11  & The Client API shall handle the low-level interaction with the Market Agent and Simulation Agent and provide callback methods for receiving messages. & M \\
         
\end{longtable}
\end{center}

\needspace{11\baselineskip}
\subsection{Non Functional Requirements}

\begin{table}[htbp]
\begin{center}
\begin{longtable}{c p{4.2in} c }

\multicolumn{1}{c}{\textbf{ID}}           &
\multicolumn{1}{c}{\textbf{Requirement}}  &
\multicolumn{1}{c}{\textbf{Priority}}     \\        
\toprule

N01  & The System shall maintain $\geq70\%$ unit test coverage.    & M \\
N02  & The System shall be integration tested.                     & M \\
N03  & The System shall be highly modular and decoupled.           & M \\
N04  & The System shall be very well javadoced.                    & M \\
N05  & The System shall be implemented on top of the JADE Framework.   & M \\
N06  & The System shall closely approximate message types and formats of industry standard messaging formats. & C \\
 
\end{longtable}
\end{center}
\end{table}



\chapter{Implementation}
\label{Chapters/Implementation}

\begin{itemize}
\item Describe the design of what you have created.
\item Start with the application architecture, giving its overall structure and the components that make up that structure.
\item Give a description of the design of each of the the components that make up the architecture.
\item Include the database or storage representation.
\item Provide implementation details as necessary.
As with other chapters, the structure and contents of this chapter will depend on the nature of your project, so the list above is only a suggestion not a fixed requirement.
\end{itemize}


Find an ordering and form of words so that the design is clear, focusing on the interesting design decisions. For example, what were the alternatiect one particular solution? You have a limited number of pages so be selective about details. Also remember that someone (your examiners!) has to read this so don’t overwhelm them with intricate descriptions of everything that only you can follow – but do make sure the key details of the solution are in place. Use appropriate terminology and demonstrate that you have a good understanding of the Computer Science principles involved.

You can use diagrams and screen shots to help explain the design but don’t overuse them. Diagrams and screen shots should add information, not duplicate what is written in the text, and definitely avoid page after page of diagrams as this will disrupt the flow of your text. Where relevant, UML diagrams can certainly be used but, again, don’t flood the chapter with diagrams. Additional diagrams can always be included in an appendix section.

It may be useful to include sections of code to highlight how a particular algorithm is implemented or how an interesting programming problem was solved. However, avoid lengthy sections of code, as they can disrupt the flow of the text. Also make sure that your code fragments are readable, easy to follow and properly laid out. It may be better to use pseudo-code rather than actual code, especially when describing an algorithm. If you need to make use of longer sections of code, you can put the code in the appendix and reference it from the text.
An alternative way to organise the content of both this chapter and the preceding one, suitable for some projects, is to have a sequence of chapters for each major iteration of the project. This allows the progression of the project to be shown, with each iteration building on the last.
This is a core chapter in your report and will usually be quite substantial, 10 pages or more.

\section{Things to cover}
\begin{itemize}
\item Modules.
\item Logging.
\item Ontologies.
\end{itemize}

\section{Things to draw attention to}
\begin{itemize}
\item The principle of defensive programming.
\item Separation of concerns with extensive use of programming to an interface.
\item Information hiding with use of private packages (OSGi) that contain implementations;
\item Hiding implementations behind factories.
\item Modularised design.
\item Fluent APIs.
\item Extensive unit and integration testing.
\item jade-unit.
\item mocking.
\item Immutable classes.
\item Careful consideration of threading issues and hence multithreaded design.
\item Highlight the fact that I needed to work around the problems of testing, etc.
\item Focus on how problems were anticipated, so they were planned.

\end{itemize}


\chapter{Testing and Validation}
\label{Chapters/Testing-and-Validation}

\begin{itemize}
\item Describe your testing strategy (unit, functional, acceptance testing
and how they are carried out). How were test cases selected.
\item Examples of specific tests and how they were carried out (e.g., using mock objects to break dependencies). Focus on the interesting cases.
\item A summary of the test results and what coverage was achieved. The detailed test report(s) should appear in the appendix.
\end{itemize}

If your project requires substantial evaluation of data and results, or other forms of testing that are not code-based, then adapt this chapter to suit.
This chapter will typically be 2-4 pages in length but could be more de- pending on the depth of testing done.

\chapter{Conclusions and Further Work}
\label{Chapters/Conclusion-and-Further-Work}

\begin{itemize}
\item A summary of what the project has achieved. Address each goal set
out in the introduction.
\item A critical evaluation of the results of the project (e.g., how well were the goals met, is the application fit for purpose, has good design and implementation practice been followed, was the right implementation technology chosen and so on).
\item Future work. How could the project be developed if you had another 6 months.
\item Wrap-up and final thoughts on your project.
\end{itemize}

This chapter is typically 2-4 pages long but could be longer if the project
work requires more extensive evaluation.

%\end{onehalfspace}

%\begin{singlespace}
%\begin{footnotesize}
%\begin{twocolumn}
\bibliographystyle{plainnat}
%\bibliographystyle{IEEEtran}
\bibliography{Bibliography}
%\end{twocolumn}
%\end{footnotesize}
%\end{singlespace}

\end{document}
