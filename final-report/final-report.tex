\documentclass[ openright,titlepage,numbers=noenddot,headinclude,%twoside,
                footinclude=true,BCOR=5mm,paper=a4,fontsize=12pt,a4paper,english%
                ]{scrreprt}
% ****************************************************************************************************
% classicthesis-config.tex 
% formerly known as loadpackages.sty, classicthesis-ldpkg.sty, and classicthesis-preamble.sty 
% Use it at the beginning of your ClassicThesis.tex, or as a LaTeX Preamble 
% in your ClassicThesis.{tex,lyx} with % ****************************************************************************************************
% classicthesis-config.tex 
% formerly known as loadpackages.sty, classicthesis-ldpkg.sty, and classicthesis-preamble.sty 
% Use it at the beginning of your ClassicThesis.tex, or as a LaTeX Preamble 
% in your ClassicThesis.{tex,lyx} with % ****************************************************************************************************
% classicthesis-config.tex 
% formerly known as loadpackages.sty, classicthesis-ldpkg.sty, and classicthesis-preamble.sty 
% Use it at the beginning of your ClassicThesis.tex, or as a LaTeX Preamble 
% in your ClassicThesis.{tex,lyx} with \input{classicthesis-config}
% ****************************************************************************************************  
% If you like the classicthesis, then I would appreciate a postca the postcards I received so far is available online at 
% http://postcards.miede.de
% ****************************************************************************************************

% ****************************************************************************************************
% 1. Configure classicthesis for your needs here, e.g., remove "drafting" below 
% in order to deactivate the time-stamp on the pages
% ****************************************************************************************************
\PassOptionsToPackage{eulerchapternumbers,listings,drafting,dottedtoc
				 pdfspacing,floatperchapter,linedheaders,
				 subfig,beramono,eulermath,parts}{classicthesis}										
% ********************************************************************
% Available options for classicthesis.sty 
% (see ClassicThesis.pdf for more information):
% drafting
% parts nochapters linedheaders
% eulerchapternumbers beramono eulermath pdfspacing minionprospacing
% tocaligned dottedtoc manychapters
% listings floatperchapter subfig
% ********************************************************************

% ********************************************************************
% Triggers for this config
% ******************************************************************** 
\usepackage{ifthen}
\newboolean{enable-backrefs} % enable backrefs in the bibliography
\setboolean{enable-backrefs}{false} % true false
% ****************************************************************************************************


% ****************************************************************************************************
% 2. Personal data and user ad-hoc commands
% ****************************************************************************************************
\newcommand{\myTitle}{A Classic Thesis Style\xspace}
\newcommand{\mySubtitle}{An Homage to The Elements of Typographic Style\xspace}
\newcommand{\myDegree}{Doktor-Ingenieur (Dr.-Ing.)\xspace}
\newcommand{\myName}{Andr\'e Miede\xspace}
\newcommand{\myProf}{Put name here\xspace}
\newcommand{\myOtherProf}{Put name here\xspace}
\newcommand{\mySupervisor}{Put name here\xspace}
\newcommand{\myFaculty}{Put data here\xspace}
\newcommand{\myDepartment}{Put data here\xspace}
\newcommand{\myUni}{Put data here\xspace}
\newcommand{\myLocation}{Darmstadt\xspace}
\newcommand{\myTime}{December 2011\xspace}
\newcommand{\myVersion}{version 4.0\xspace}

% ********************************************************************
% Setup, finetuning, and useful commands
% ********************************************************************
\newcounter{dummy} % necessary for correct hyperlinks (to index, bib, etc.)
\newlength{\abcd} % for ab..z string length calculation
\providecommand{\mLyX}{L\kern-.1667em\lower.25em\hbox{Y}\kern-.125emX\@}
\newcommand{\ie}{i.\,e.}
\newcommand{\Ie}{I.\,e.}
\newcommand{\eg}{e.\,g.}
\newcommand{\Eg}{E.\,g.} 
% ****************************************************************************************************


% ****************************************************************************************************
% 3. Loading some handy packages
% ****************************************************************************************************
% ******************************************************************** 
% Packages with options that might require adjustments
% ******************************************************************** 
\PassOptionsToPackage{utf8}{inputenc}	% latin9 (ISO-8859-9) = latin1+"Euro sign"
 \usepackage{inputenc}				

%\PassOptionsToPackage{ngerman,american}{babel}   % change this to your language(s)
% Spanish languages need extra options in order to work with this template
%\PassOptionsToPackage{spanish,es-lcroman}{babel}
 \usepackage{babel}					

\PassOptionsToPackage{square,numbers}{natbib}
 \usepackage{natbib}				

\PassOptionsToPackage{fleqn}{amsmath}		% math environments and more by the AMS 
 \usepackage{amsmath}

% ******************************************************************** 
% General useful packages
% ******************************************************************** 
\PassOptionsToPackage{T1}{fontenc} % T2A for cyrillics
	\usepackage{fontenc}                 
\usepackage{xspace} % to get the spacing after macros right  
\usepackage{mparhack} % get marginpar right
\usepackage{fixltx2e} % fixes some LaTeX stuff 
\PassOptionsToPackage{printonlyused,smaller}{acronym}
	\usepackage{acronym} % nice macros for handling all acronyms in the thesis
%\renewcommand*{\acsfont}[1]{\textssc{#1}} % for MinionPro
\renewcommand{\bflabel}[1]{{#1}\hfill} % fix the list of acronyms
% ****************************************************************************************************




% ****************************************************************************************************
% 4. Setup floats: tables, (sub)figures, and captions
% ****************************************************************************************************
\usepackage{tabularx} % better tables
	\setlength{\extrarowheight}{3pt} % increase table row height
\newcommand{\tableheadline}[1]{\multicolumn{1}{c}{\spacedlowsmallcaps{#1}}}
\newcommand{\myfloatalign}{\centering} % to be used with each float for alignment
\usepackage{caption}
\captionsetup{format=hang,font=small}
\usepackage{subfig}  
% ****************************************************************************************************


% ****************************************************************************************************
% 5. Setup code listings
% ****************************************************************************************************
\usepackage{listings} 
%\lstset{emph={trueIndex,root},emphstyle=\color{BlueViolet}}%\underbar} % for special keywords
\lstset{language=[LaTeX]Tex,%C++,
    keywordstyle=\color{RoyalBlue},%\bfseries,
    basicstyle=\small\ttfamily,
    %identifierstyle=\color{NavyBlue},
    commentstyle=\color{Green}\ttfamily,
    stringstyle=\rmfamily,
    numbers=none,%left,%
    numberstyle=\scriptsize,%\tiny
    stepnumber=5,
    numbersep=8pt,
    showstringspaces=false,
    breaklines=true,
    frameround=ftff,
    frame=single,
    belowcaptionskip=.75\baselineskip
    %frame=L
} 
% ****************************************************************************************************    		   


% ****************************************************************************************************
% 6. PDFLaTeX, hyperreferences and citation backreferences
% ****************************************************************************************************
% ********************************************************************
% Using PDFLaTeX
% ********************************************************************
\PassOptionsToPackage{pdftex,hyperfootnotes=false,pdfpagelabels}{hyperref}
	\usepackage{hyperref}  % backref linktocpage pagebackref
\pdfcompresslevel=9
\pdfadjustspacing=1 
\PassOptionsToPackage{pdftex}{graphicx}
	\usepackage{graphicx} 

% ********************************************************************
% Setup the style of the backrefs from the bibliography
% (translate the options to any language you use)
% ********************************************************************
\newcommand{\backrefnotcitedstring}{\relax}%(Not cited.)
\newcommand{\backrefcitedsinglestring}[1]{(Cited on page~#1.)}
\newcommand{\backrefcitedmultistring}[1]{(Cited on pages~#1.)}
\ifthenelse{\boolean{enable-backrefs}}%
{%
		\PassOptionsToPackage{hyperpageref}{backref}
		\usepackage{backref} % to be loaded after hyperref package 
		   \renewcommand{\backreftwosep}{ and~} % separate 2 pages
		   \renewcommand{\backreflastsep}{, and~} % separate last of longer list
		   \renewcommand*{\backref}[1]{}  % disable standard
		   \renewcommand*{\backrefalt}[4]{% detailed backref
		      \ifcase #1 %
		         \backrefnotcitedstring%
		      \or%
		         \backrefcitedsinglestring{#2}%
		      \else%
		         \backrefcitedmultistring{#2}%
		      \fi}%
}{\relax}    

% ********************************************************************
% Hyperreferences
% ********************************************************************
\hypersetup{%
    %draft,	% = no hyperlinking at all (useful in b/w printouts)
    colorlinks=true, linktocpage=true, pdfstartpage=3, pdfstartview=FitV,%
    % uncomment the following line if you want to have black links (e.g., for printing)
    %colorlinks=false, linktocpage=false, pdfborder={0 0 0}, pdfstartpage=3, pdfstartview=FitV,% 
    breaklinks=true, pdfpagemode=UseNone, pageanchor=true, pdfpagemode=UseOutlines,%
    plainpages=false, bookmarksnumbered, bookmarksopen=true, bookmarksopenlevel=1,%
    hypertexnames=true, pdfhighlight=/O,%nesting=true,%frenchlinks,%
    urlcolor=webbrown, linkcolor=RoyalBlue, citecolor=webgreen, %pagecolor=RoyalBlue,%
    %urlcolor=Black, linkcolor=Black, citecolor=Black, %pagecolor=Black,%
    pdftitle={\myTitle},%
    pdfauthor={\textcopyright\ \myName, \myUni, \myFaculty},%
    pdfsubject={},%
    pdfkeywords={},%
    pdfcreator={pdfLaTeX},%
    pdfproducer={LaTeX with hyperref and classicthesis}%
}   

% ********************************************************************
% Setup autoreferences
% ********************************************************************
% There are some issues regarding autorefnames
% http://www.ureader.de/msg/136221647.aspx
% http://www.tex.ac.uk/cgi-bin/texfaq2html?label=latexwords
% you have to redefine the makros for the 
% language you use, e.g., american, ngerman
% (as chosen when loading babel/AtBeginDocument)
% ********************************************************************
\makeatletter
\@ifpackageloaded{babel}%
    {%
       \addto\extrasamerican{%
					\renewcommand*{\figureautorefname}{Figure}%
					\renewcommand*{\tableautorefname}{Table}%
					\renewcommand*{\partautorefname}{Part}%
					\renewcommand*{\chapterautorefname}{Chapter}%
					\renewcommand*{\sectionautorefname}{Section}%
					\renewcommand*{\subsectionautorefname}{Section}%
					\renewcommand*{\subsubsectionautorefname}{Section}% 	
				}%
       \addto\extrasngerman{% 
					\renewcommand*{\paragraphautorefname}{Absatz}%
					\renewcommand*{\subparagraphautorefname}{Unterabsatz}%
					\renewcommand*{\footnoteautorefname}{Fu\"snote}%
					\renewcommand*{\FancyVerbLineautorefname}{Zeile}%
					\renewcommand*{\theoremautorefname}{Theorem}%
					\renewcommand*{\appendixautorefname}{Anhang}%
					\renewcommand*{\equationautorefname}{Gleichung}%        
					\renewcommand*{\itemautorefname}{Punkt}%
				}%
			% Fix to getting autorefs for subfigures right (thanks to Belinda Vogt for changing the definition)
			\providecommand{\subfigureautorefname}{\figureautorefname}%  			
    }{\relax}
\makeatother


% ****************************************************************************************************
% 7. Last calls before the bar closes
% ****************************************************************************************************
% ********************************************************************
% Development Stuff
% ********************************************************************
\listfiles
%\PassOptionsToPackage{l2tabu,orthodox,abort}{nag}
%	\usepackage{nag}
%\PassOptionsToPackage{warning, all}{onlyamsmath}
%	\usepackage{onlyamsmath}

% ********************************************************************
% Last, but not least...
% ********************************************************************
\usepackage{classicthesis} 

% Use arsclassica as well
\usepackage[dottedtoc]{arsclassica}
% ****************************************************************************************************


% ****************************************************************************************************
% 8. Further adjustments (experimental)
% ****************************************************************************************************
% ********************************************************************
% Changing the text area
% ********************************************************************
%\linespread{1.05} % a bit more for Palatino
%\areaset[current]{312pt}{761pt} % 686 (factor 2.2) + 33 head + 42 head \the\footskip
%\setlength{\marginparwidth}{7em}%
%\setlength{\marginparsep}{2em}%

% ********************************************************************
% Using different fonts
% ********************************************************************
%\usepackage[oldstylenums]{kpfonts} % oldstyle notextcomp
%\usepackage[osf]{libertine}
%\usepackage{hfoldsty} % Computer Modern with osf
%\usepackage[light,condensed,math]{iwona}
%\renewcommand{\sfdefault}{iwona}
%\usepackage{lmodern} % <-- no osf support :-(
%\usepackage[urw-garamond]{mathdesign} <-- no osf support :-(
% ****************************************************************************************************

% ****************************************************************************************************
% 9. Packages enabled by me
% ****************************************************************************************************
\usepackage{listings}
\usepackage{todonotes}
\usepackage{fancyhdr}
\usepackage[utf8]{fontenc}
\usepackage{babel}
\usepackage{booktabs}
\usepackage{cleveref}
\usepackage{setspace}
\usepackage{longtable}
\usepackage{needspace}

\usepackage[top=1.5in, bottom=1.2in, left=1in, right=1in]{geometry}

\usepackage{tikz}
\usetikzlibrary{shapes,arrows}
\usepackage{bigints}
\usetikzlibrary{shapes.geometric,shapes.arrows,arrows,backgrounds,fit,positioning,shapes.symbols,chains,decorations.pathmorphing}
%\usetikzlibrary{matrix,chains,scopes,positioning,arrows,fit}

\usepackage{amsmath, amssymb, amsthm}
\usepackage[T1]{fontenc}

% Set margins
%\addtolength{\oddsidemargin}{-.875in}
%\addtolength{\evensidemargin}{-.875in}
%\addtolength{\textwidth}{1.75in}

%\addtolength{\topmargin}{-.875in}
%\addtolength{\textheight}{1.75in}

% ****************************************************************************************************  
% If you like the classicthesis, then I would appreciate a postca the postcards I received so far is available online at 
% http://postcards.miede.de
% ****************************************************************************************************

% ****************************************************************************************************
% 1. Configure classicthesis for your needs here, e.g., remove "drafting" below 
% in order to deactivate the time-stamp on the pages
% ****************************************************************************************************
\PassOptionsToPackage{eulerchapternumbers,listings,drafting,dottedtoc
				 pdfspacing,floatperchapter,linedheaders,
				 subfig,beramono,eulermath,parts}{classicthesis}										
% ********************************************************************
% Available options for classicthesis.sty 
% (see ClassicThesis.pdf for more information):
% drafting
% parts nochapters linedheaders
% eulerchapternumbers beramono eulermath pdfspacing minionprospacing
% tocaligned dottedtoc manychapters
% listings floatperchapter subfig
% ********************************************************************

% ********************************************************************
% Triggers for this config
% ******************************************************************** 
\usepackage{ifthen}
\newboolean{enable-backrefs} % enable backrefs in the bibliography
\setboolean{enable-backrefs}{false} % true false
% ****************************************************************************************************


% ****************************************************************************************************
% 2. Personal data and user ad-hoc commands
% ****************************************************************************************************
\newcommand{\myTitle}{A Classic Thesis Style\xspace}
\newcommand{\mySubtitle}{An Homage to The Elements of Typographic Style\xspace}
\newcommand{\myDegree}{Doktor-Ingenieur (Dr.-Ing.)\xspace}
\newcommand{\myName}{Andr\'e Miede\xspace}
\newcommand{\myProf}{Put name here\xspace}
\newcommand{\myOtherProf}{Put name here\xspace}
\newcommand{\mySupervisor}{Put name here\xspace}
\newcommand{\myFaculty}{Put data here\xspace}
\newcommand{\myDepartment}{Put data here\xspace}
\newcommand{\myUni}{Put data here\xspace}
\newcommand{\myLocation}{Darmstadt\xspace}
\newcommand{\myTime}{December 2011\xspace}
\newcommand{\myVersion}{version 4.0\xspace}

% ********************************************************************
% Setup, finetuning, and useful commands
% ********************************************************************
\newcounter{dummy} % necessary for correct hyperlinks (to index, bib, etc.)
\newlength{\abcd} % for ab..z string length calculation
\providecommand{\mLyX}{L\kern-.1667em\lower.25em\hbox{Y}\kern-.125emX\@}
\newcommand{\ie}{i.\,e.}
\newcommand{\Ie}{I.\,e.}
\newcommand{\eg}{e.\,g.}
\newcommand{\Eg}{E.\,g.} 
% ****************************************************************************************************


% ****************************************************************************************************
% 3. Loading some handy packages
% ****************************************************************************************************
% ******************************************************************** 
% Packages with options that might require adjustments
% ******************************************************************** 
\PassOptionsToPackage{utf8}{inputenc}	% latin9 (ISO-8859-9) = latin1+"Euro sign"
 \usepackage{inputenc}				

%\PassOptionsToPackage{ngerman,american}{babel}   % change this to your language(s)
% Spanish languages need extra options in order to work with this template
%\PassOptionsToPackage{spanish,es-lcroman}{babel}
 \usepackage{babel}					

\PassOptionsToPackage{square,numbers}{natbib}
 \usepackage{natbib}				

\PassOptionsToPackage{fleqn}{amsmath}		% math environments and more by the AMS 
 \usepackage{amsmath}

% ******************************************************************** 
% General useful packages
% ******************************************************************** 
\PassOptionsToPackage{T1}{fontenc} % T2A for cyrillics
	\usepackage{fontenc}                 
\usepackage{xspace} % to get the spacing after macros right  
\usepackage{mparhack} % get marginpar right
\usepackage{fixltx2e} % fixes some LaTeX stuff 
\PassOptionsToPackage{printonlyused,smaller}{acronym}
	\usepackage{acronym} % nice macros for handling all acronyms in the thesis
%\renewcommand*{\acsfont}[1]{\textssc{#1}} % for MinionPro
\renewcommand{\bflabel}[1]{{#1}\hfill} % fix the list of acronyms
% ****************************************************************************************************




% ****************************************************************************************************
% 4. Setup floats: tables, (sub)figures, and captions
% ****************************************************************************************************
\usepackage{tabularx} % better tables
	\setlength{\extrarowheight}{3pt} % increase table row height
\newcommand{\tableheadline}[1]{\multicolumn{1}{c}{\spacedlowsmallcaps{#1}}}
\newcommand{\myfloatalign}{\centering} % to be used with each float for alignment
\usepackage{caption}
\captionsetup{format=hang,font=small}
\usepackage{subfig}  
% ****************************************************************************************************


% ****************************************************************************************************
% 5. Setup code listings
% ****************************************************************************************************
\usepackage{listings} 
%\lstset{emph={trueIndex,root},emphstyle=\color{BlueViolet}}%\underbar} % for special keywords
\lstset{language=[LaTeX]Tex,%C++,
    keywordstyle=\color{RoyalBlue},%\bfseries,
    basicstyle=\small\ttfamily,
    %identifierstyle=\color{NavyBlue},
    commentstyle=\color{Green}\ttfamily,
    stringstyle=\rmfamily,
    numbers=none,%left,%
    numberstyle=\scriptsize,%\tiny
    stepnumber=5,
    numbersep=8pt,
    showstringspaces=false,
    breaklines=true,
    frameround=ftff,
    frame=single,
    belowcaptionskip=.75\baselineskip
    %frame=L
} 
% ****************************************************************************************************    		   


% ****************************************************************************************************
% 6. PDFLaTeX, hyperreferences and citation backreferences
% ****************************************************************************************************
% ********************************************************************
% Using PDFLaTeX
% ********************************************************************
\PassOptionsToPackage{pdftex,hyperfootnotes=false,pdfpagelabels}{hyperref}
	\usepackage{hyperref}  % backref linktocpage pagebackref
\pdfcompresslevel=9
\pdfadjustspacing=1 
\PassOptionsToPackage{pdftex}{graphicx}
	\usepackage{graphicx} 

% ********************************************************************
% Setup the style of the backrefs from the bibliography
% (translate the options to any language you use)
% ********************************************************************
\newcommand{\backrefnotcitedstring}{\relax}%(Not cited.)
\newcommand{\backrefcitedsinglestring}[1]{(Cited on page~#1.)}
\newcommand{\backrefcitedmultistring}[1]{(Cited on pages~#1.)}
\ifthenelse{\boolean{enable-backrefs}}%
{%
		\PassOptionsToPackage{hyperpageref}{backref}
		\usepackage{backref} % to be loaded after hyperref package 
		   \renewcommand{\backreftwosep}{ and~} % separate 2 pages
		   \renewcommand{\backreflastsep}{, and~} % separate last of longer list
		   \renewcommand*{\backref}[1]{}  % disable standard
		   \renewcommand*{\backrefalt}[4]{% detailed backref
		      \ifcase #1 %
		         \backrefnotcitedstring%
		      \or%
		         \backrefcitedsinglestring{#2}%
		      \else%
		         \backrefcitedmultistring{#2}%
		      \fi}%
}{\relax}    

% ********************************************************************
% Hyperreferences
% ********************************************************************
\hypersetup{%
    %draft,	% = no hyperlinking at all (useful in b/w printouts)
    colorlinks=true, linktocpage=true, pdfstartpage=3, pdfstartview=FitV,%
    % uncomment the following line if you want to have black links (e.g., for printing)
    %colorlinks=false, linktocpage=false, pdfborder={0 0 0}, pdfstartpage=3, pdfstartview=FitV,% 
    breaklinks=true, pdfpagemode=UseNone, pageanchor=true, pdfpagemode=UseOutlines,%
    plainpages=false, bookmarksnumbered, bookmarksopen=true, bookmarksopenlevel=1,%
    hypertexnames=true, pdfhighlight=/O,%nesting=true,%frenchlinks,%
    urlcolor=webbrown, linkcolor=RoyalBlue, citecolor=webgreen, %pagecolor=RoyalBlue,%
    %urlcolor=Black, linkcolor=Black, citecolor=Black, %pagecolor=Black,%
    pdftitle={\myTitle},%
    pdfauthor={\textcopyright\ \myName, \myUni, \myFaculty},%
    pdfsubject={},%
    pdfkeywords={},%
    pdfcreator={pdfLaTeX},%
    pdfproducer={LaTeX with hyperref and classicthesis}%
}   

% ********************************************************************
% Setup autoreferences
% ********************************************************************
% There are some issues regarding autorefnames
% http://www.ureader.de/msg/136221647.aspx
% http://www.tex.ac.uk/cgi-bin/texfaq2html?label=latexwords
% you have to redefine the makros for the 
% language you use, e.g., american, ngerman
% (as chosen when loading babel/AtBeginDocument)
% ********************************************************************
\makeatletter
\@ifpackageloaded{babel}%
    {%
       \addto\extrasamerican{%
					\renewcommand*{\figureautorefname}{Figure}%
					\renewcommand*{\tableautorefname}{Table}%
					\renewcommand*{\partautorefname}{Part}%
					\renewcommand*{\chapterautorefname}{Chapter}%
					\renewcommand*{\sectionautorefname}{Section}%
					\renewcommand*{\subsectionautorefname}{Section}%
					\renewcommand*{\subsubsectionautorefname}{Section}% 	
				}%
       \addto\extrasngerman{% 
					\renewcommand*{\paragraphautorefname}{Absatz}%
					\renewcommand*{\subparagraphautorefname}{Unterabsatz}%
					\renewcommand*{\footnoteautorefname}{Fu\"snote}%
					\renewcommand*{\FancyVerbLineautorefname}{Zeile}%
					\renewcommand*{\theoremautorefname}{Theorem}%
					\renewcommand*{\appendixautorefname}{Anhang}%
					\renewcommand*{\equationautorefname}{Gleichung}%        
					\renewcommand*{\itemautorefname}{Punkt}%
				}%
			% Fix to getting autorefs for subfigures right (thanks to Belinda Vogt for changing the definition)
			\providecommand{\subfigureautorefname}{\figureautorefname}%  			
    }{\relax}
\makeatother


% ****************************************************************************************************
% 7. Last calls before the bar closes
% ****************************************************************************************************
% ********************************************************************
% Development Stuff
% ********************************************************************
\listfiles
%\PassOptionsToPackage{l2tabu,orthodox,abort}{nag}
%	\usepackage{nag}
%\PassOptionsToPackage{warning, all}{onlyamsmath}
%	\usepackage{onlyamsmath}

% ********************************************************************
% Last, but not least...
% ********************************************************************
\usepackage{classicthesis} 

% Use arsclassica as well
\usepackage[dottedtoc]{arsclassica}
% ****************************************************************************************************


% ****************************************************************************************************
% 8. Further adjustments (experimental)
% ****************************************************************************************************
% ********************************************************************
% Changing the text area
% ********************************************************************
%\linespread{1.05} % a bit more for Palatino
%\areaset[current]{312pt}{761pt} % 686 (factor 2.2) + 33 head + 42 head \the\footskip
%\setlength{\marginparwidth}{7em}%
%\setlength{\marginparsep}{2em}%

% ********************************************************************
% Using different fonts
% ********************************************************************
%\usepackage[oldstylenums]{kpfonts} % oldstyle notextcomp
%\usepackage[osf]{libertine}
%\usepackage{hfoldsty} % Computer Modern with osf
%\usepackage[light,condensed,math]{iwona}
%\renewcommand{\sfdefault}{iwona}
%\usepackage{lmodern} % <-- no osf support :-(
%\usepackage[urw-garamond]{mathdesign} <-- no osf support :-(
% ****************************************************************************************************

% ****************************************************************************************************
% 9. Packages enabled by me
% ****************************************************************************************************
\usepackage{listings}
\usepackage{todonotes}
\usepackage{fancyhdr}
\usepackage[utf8]{fontenc}
\usepackage{babel}
\usepackage{booktabs}
\usepackage{cleveref}
\usepackage{setspace}
\usepackage{longtable}
\usepackage{needspace}

\usepackage[top=1.5in, bottom=1.2in, left=1in, right=1in]{geometry}

\usepackage{tikz}
\usetikzlibrary{shapes,arrows}
\usepackage{bigints}
\usetikzlibrary{shapes.geometric,shapes.arrows,arrows,backgrounds,fit,positioning,shapes.symbols,chains,decorations.pathmorphing}
%\usetikzlibrary{matrix,chains,scopes,positioning,arrows,fit}

\usepackage{amsmath, amssymb, amsthm}
\usepackage[T1]{fontenc}

% Set margins
%\addtolength{\oddsidemargin}{-.875in}
%\addtolength{\evensidemargin}{-.875in}
%\addtolength{\textwidth}{1.75in}

%\addtolength{\topmargin}{-.875in}
%\addtolength{\textheight}{1.75in}

% ****************************************************************************************************  
% If you like the classicthesis, then I would appreciate a postca the postcards I received so far is available online at 
% http://postcards.miede.de
% ****************************************************************************************************

% ****************************************************************************************************
% 1. Configure classicthesis for your needs here, e.g., remove "drafting" below 
% in order to deactivate the time-stamp on the pages
% ****************************************************************************************************
\PassOptionsToPackage{eulerchapternumbers,listings,drafting,dottedtoc
				 pdfspacing,floatperchapter,linedheaders,
				 subfig,beramono,eulermath,parts}{classicthesis}										
% ********************************************************************
% Available options for classicthesis.sty 
% (see ClassicThesis.pdf for more information):
% drafting
% parts nochapters linedheaders
% eulerchapternumbers beramono eulermath pdfspacing minionprospacing
% tocaligned dottedtoc manychapters
% listings floatperchapter subfig
% ********************************************************************

% ********************************************************************
% Triggers for this config
% ******************************************************************** 
\usepackage{ifthen}
\newboolean{enable-backrefs} % enable backrefs in the bibliography
\setboolean{enable-backrefs}{false} % true false
% ****************************************************************************************************


% ****************************************************************************************************
% 2. Personal data and user ad-hoc commands
% ****************************************************************************************************
\newcommand{\myTitle}{A Classic Thesis Style\xspace}
\newcommand{\mySubtitle}{An Homage to The Elements of Typographic Style\xspace}
\newcommand{\myDegree}{Doktor-Ingenieur (Dr.-Ing.)\xspace}
\newcommand{\myName}{Andr\'e Miede\xspace}
\newcommand{\myProf}{Put name here\xspace}
\newcommand{\myOtherProf}{Put name here\xspace}
\newcommand{\mySupervisor}{Put name here\xspace}
\newcommand{\myFaculty}{Put data here\xspace}
\newcommand{\myDepartment}{Put data here\xspace}
\newcommand{\myUni}{Put data here\xspace}
\newcommand{\myLocation}{Darmstadt\xspace}
\newcommand{\myTime}{December 2011\xspace}
\newcommand{\myVersion}{version 4.0\xspace}

% ********************************************************************
% Setup, finetuning, and useful commands
% ********************************************************************
\newcounter{dummy} % necessary for correct hyperlinks (to index, bib, etc.)
\newlength{\abcd} % for ab..z string length calculation
\providecommand{\mLyX}{L\kern-.1667em\lower.25em\hbox{Y}\kern-.125emX\@}
\newcommand{\ie}{i.\,e.}
\newcommand{\Ie}{I.\,e.}
\newcommand{\eg}{e.\,g.}
\newcommand{\Eg}{E.\,g.} 
% ****************************************************************************************************


% ****************************************************************************************************
% 3. Loading some handy packages
% ****************************************************************************************************
% ******************************************************************** 
% Packages with options that might require adjustments
% ******************************************************************** 
\PassOptionsToPackage{utf8}{inputenc}	% latin9 (ISO-8859-9) = latin1+"Euro sign"
 \usepackage{inputenc}				

%\PassOptionsToPackage{ngerman,american}{babel}   % change this to your language(s)
% Spanish languages need extra options in order to work with this template
%\PassOptionsToPackage{spanish,es-lcroman}{babel}
 \usepackage{babel}					

\PassOptionsToPackage{square,numbers}{natbib}
 \usepackage{natbib}				

\PassOptionsToPackage{fleqn}{amsmath}		% math environments and more by the AMS 
 \usepackage{amsmath}

% ******************************************************************** 
% General useful packages
% ******************************************************************** 
\PassOptionsToPackage{T1}{fontenc} % T2A for cyrillics
	\usepackage{fontenc}                 
\usepackage{xspace} % to get the spacing after macros right  
\usepackage{mparhack} % get marginpar right
\usepackage{fixltx2e} % fixes some LaTeX stuff 
\PassOptionsToPackage{printonlyused,smaller}{acronym}
	\usepackage{acronym} % nice macros for handling all acronyms in the thesis
%\renewcommand*{\acsfont}[1]{\textssc{#1}} % for MinionPro
\renewcommand{\bflabel}[1]{{#1}\hfill} % fix the list of acronyms
% ****************************************************************************************************




% ****************************************************************************************************
% 4. Setup floats: tables, (sub)figures, and captions
% ****************************************************************************************************
\usepackage{tabularx} % better tables
	\setlength{\extrarowheight}{3pt} % increase table row height
\newcommand{\tableheadline}[1]{\multicolumn{1}{c}{\spacedlowsmallcaps{#1}}}
\newcommand{\myfloatalign}{\centering} % to be used with each float for alignment
\usepackage{caption}
\captionsetup{format=hang,font=small}
\usepackage{subfig}  
% ****************************************************************************************************


% ****************************************************************************************************
% 5. Setup code listings
% ****************************************************************************************************
\usepackage{listings} 
%\lstset{emph={trueIndex,root},emphstyle=\color{BlueViolet}}%\underbar} % for special keywords
\lstset{language=[LaTeX]Tex,%C++,
    keywordstyle=\color{RoyalBlue},%\bfseries,
    basicstyle=\small\ttfamily,
    %identifierstyle=\color{NavyBlue},
    commentstyle=\color{Green}\ttfamily,
    stringstyle=\rmfamily,
    numbers=none,%left,%
    numberstyle=\scriptsize,%\tiny
    stepnumber=5,
    numbersep=8pt,
    showstringspaces=false,
    breaklines=true,
    frameround=ftff,
    frame=single,
    belowcaptionskip=.75\baselineskip
    %frame=L
} 
% ****************************************************************************************************    		   


% ****************************************************************************************************
% 6. PDFLaTeX, hyperreferences and citation backreferences
% ****************************************************************************************************
% ********************************************************************
% Using PDFLaTeX
% ********************************************************************
\PassOptionsToPackage{pdftex,hyperfootnotes=false,pdfpagelabels}{hyperref}
	\usepackage{hyperref}  % backref linktocpage pagebackref
\pdfcompresslevel=9
\pdfadjustspacing=1 
\PassOptionsToPackage{pdftex}{graphicx}
	\usepackage{graphicx} 

% ********************************************************************
% Setup the style of the backrefs from the bibliography
% (translate the options to any language you use)
% ********************************************************************
\newcommand{\backrefnotcitedstring}{\relax}%(Not cited.)
\newcommand{\backrefcitedsinglestring}[1]{(Cited on page~#1.)}
\newcommand{\backrefcitedmultistring}[1]{(Cited on pages~#1.)}
\ifthenelse{\boolean{enable-backrefs}}%
{%
		\PassOptionsToPackage{hyperpageref}{backref}
		\usepackage{backref} % to be loaded after hyperref package 
		   \renewcommand{\backreftwosep}{ and~} % separate 2 pages
		   \renewcommand{\backreflastsep}{, and~} % separate last of longer list
		   \renewcommand*{\backref}[1]{}  % disable standard
		   \renewcommand*{\backrefalt}[4]{% detailed backref
		      \ifcase #1 %
		         \backrefnotcitedstring%
		      \or%
		         \backrefcitedsinglestring{#2}%
		      \else%
		         \backrefcitedmultistring{#2}%
		      \fi}%
}{\relax}    

% ********************************************************************
% Hyperreferences
% ********************************************************************
\hypersetup{%
    %draft,	% = no hyperlinking at all (useful in b/w printouts)
    colorlinks=true, linktocpage=true, pdfstartpage=3, pdfstartview=FitV,%
    % uncomment the following line if you want to have black links (e.g., for printing)
    %colorlinks=false, linktocpage=false, pdfborder={0 0 0}, pdfstartpage=3, pdfstartview=FitV,% 
    breaklinks=true, pdfpagemode=UseNone, pageanchor=true, pdfpagemode=UseOutlines,%
    plainpages=false, bookmarksnumbered, bookmarksopen=true, bookmarksopenlevel=1,%
    hypertexnames=true, pdfhighlight=/O,%nesting=true,%frenchlinks,%
    urlcolor=webbrown, linkcolor=RoyalBlue, citecolor=webgreen, %pagecolor=RoyalBlue,%
    %urlcolor=Black, linkcolor=Black, citecolor=Black, %pagecolor=Black,%
    pdftitle={\myTitle},%
    pdfauthor={\textcopyright\ \myName, \myUni, \myFaculty},%
    pdfsubject={},%
    pdfkeywords={},%
    pdfcreator={pdfLaTeX},%
    pdfproducer={LaTeX with hyperref and classicthesis}%
}   

% ********************************************************************
% Setup autoreferences
% ********************************************************************
% There are some issues regarding autorefnames
% http://www.ureader.de/msg/136221647.aspx
% http://www.tex.ac.uk/cgi-bin/texfaq2html?label=latexwords
% you have to redefine the makros for the 
% language you use, e.g., american, ngerman
% (as chosen when loading babel/AtBeginDocument)
% ********************************************************************
\makeatletter
\@ifpackageloaded{babel}%
    {%
       \addto\extrasamerican{%
					\renewcommand*{\figureautorefname}{Figure}%
					\renewcommand*{\tableautorefname}{Table}%
					\renewcommand*{\partautorefname}{Part}%
					\renewcommand*{\chapterautorefname}{Chapter}%
					\renewcommand*{\sectionautorefname}{Section}%
					\renewcommand*{\subsectionautorefname}{Section}%
					\renewcommand*{\subsubsectionautorefname}{Section}% 	
				}%
       \addto\extrasngerman{% 
					\renewcommand*{\paragraphautorefname}{Absatz}%
					\renewcommand*{\subparagraphautorefname}{Unterabsatz}%
					\renewcommand*{\footnoteautorefname}{Fu\"snote}%
					\renewcommand*{\FancyVerbLineautorefname}{Zeile}%
					\renewcommand*{\theoremautorefname}{Theorem}%
					\renewcommand*{\appendixautorefname}{Anhang}%
					\renewcommand*{\equationautorefname}{Gleichung}%        
					\renewcommand*{\itemautorefname}{Punkt}%
				}%
			% Fix to getting autorefs for subfigures right (thanks to Belinda Vogt for changing the definition)
			\providecommand{\subfigureautorefname}{\figureautorefname}%  			
    }{\relax}
\makeatother


% ****************************************************************************************************
% 7. Last calls before the bar closes
% ****************************************************************************************************
% ********************************************************************
% Development Stuff
% ********************************************************************
\listfiles
%\PassOptionsToPackage{l2tabu,orthodox,abort}{nag}
%	\usepackage{nag}
%\PassOptionsToPackage{warning, all}{onlyamsmath}
%	\usepackage{onlyamsmath}

% ********************************************************************
% Last, but not least...
% ********************************************************************
\usepackage{classicthesis} 

% Use arsclassica as well
\usepackage[dottedtoc]{arsclassica}
% ****************************************************************************************************


% ****************************************************************************************************
% 8. Further adjustments (experimental)
% ****************************************************************************************************
% ********************************************************************
% Changing the text area
% ********************************************************************
%\linespread{1.05} % a bit more for Palatino
%\areaset[current]{312pt}{761pt} % 686 (factor 2.2) + 33 head + 42 head \the\footskip
%\setlength{\marginparwidth}{7em}%
%\setlength{\marginparsep}{2em}%

% ********************************************************************
% Using different fonts
% ********************************************************************
%\usepackage[oldstylenums]{kpfonts} % oldstyle notextcomp
%\usepackage[osf]{libertine}
%\usepackage{hfoldsty} % Computer Modern with osf
%\usepackage[light,condensed,math]{iwona}
%\renewcommand{\sfdefault}{iwona}
%\usepackage{lmodern} % <-- no osf support :-(
%\usepackage[urw-garamond]{mathdesign} <-- no osf support :-(
% ****************************************************************************************************

% ****************************************************************************************************
% 9. Packages enabled by me
% ****************************************************************************************************
\usepackage{listings}
\usepackage{todonotes}
\usepackage{fancyhdr}
\usepackage[utf8]{fontenc}
\usepackage{babel}
\usepackage{booktabs}
\usepackage{cleveref}
\usepackage{setspace}
\usepackage{longtable}
\usepackage{needspace}

\usepackage[top=1.5in, bottom=1.2in, left=1in, right=1in]{geometry}

\usepackage{tikz}
\usetikzlibrary{shapes,arrows}
\usepackage{bigints}
\usetikzlibrary{shapes.geometric,shapes.arrows,arrows,backgrounds,fit,positioning,shapes.symbols,chains,decorations.pathmorphing}
%\usetikzlibrary{matrix,chains,scopes,positioning,arrows,fit}

\usepackage{amsmath, amssymb, amsthm}
\usepackage[T1]{fontenc}

% Set margins
%\addtolength{\oddsidemargin}{-.875in}
%\addtolength{\evensidemargin}{-.875in}
%\addtolength{\textwidth}{1.75in}

%\addtolength{\topmargin}{-.875in}
%\addtolength{\textheight}{1.75in}


\begin{document}


\title{
\Large{\textbf{Eugene: Agent-Based Market Simulator for use in Validation and System Testing of Trading Algorithms}}\\[1.2cm]
\normalsize{Jakub Koz\l{}owski \\
Internal Supervisor: Dr. Christopher D. Clack \\
External Supervisor: Ian Rose-Miller, UBS \\[1.2cm]
Department of Computer Science \\University College London \\[1cm]}
\small{April 2011}
}
\author{} \date{}

\maketitle
\tableofcontents
\chapter*{Abstract}
At 3:40 PM on November 14\textsuperscript{th}, 2007, a buggy Credit Suisse proprietary algorithm (SmartWB) sent approximately 600,000 cancel/replace messages for non existent orders, triggerring 450,000 error messages from the NYSE and leading to a disruption of trading, with 131,000 messages frozen in queue that could not be processed and ultimately had to be deleted. Credit Suisse was fined \$150,000.~\cite{Nyse2009} On May 6, 2010, U.S. stock market has experienced a sudden price drop of 5\%, followed by a rapid recovery, in the mean time sending Accenture share price to a single cent and Sotheby’s to \$99,999.99, all in the course of about 30 minutes. Subsequent analyses of the incident concluded that the the incident was triggered by a large sell program that lead to "hot-potato" effect, where in the 14-second period more than 27,000 futures contracts were bought and sold, with the net effect of only around 200 contracts.~\cite{Kirilenko2011}

These prominent examples of software errors in Trading Algorithms highlight an urgent issue with the way these systems are tested. In order to prevent Trading Algorithms from exhibiting such behaviour, there is a need for a strategy of rigorous testing in a realistic environment. Current testing techniques (backtesting on historical data) fail to capture the dynamic nature of markets and hence may fail to identify flaws in the Trading Algorithms. In this thesis we demonstrate an Agent-Based Stock Market Simulator and evaluate its efficacy in automatic discovery of software errors in Trading Algorithms.
\vfill

\pagestyle{scrheadings}
%\doublespacing

\chapter{Introduction}
\label{Chapters/Introduction}
\begin{itemize}
\item Outline the problem you are working on, why it is interesting and
what the challenges are.
\item List your aims and goals. An aim is something you intend to achieve (e.g., learn a new programming language and apply it in solving the problem), while a goal is something specific you expect to deliver (e.g., a working application with a particular set of features).
\item Give an overview of how you carried out the project (e.g., an iterative approach).
\item A brief overview of the rest of the chapters in the report (a guide to the reader of the overall structure of the report).
\end{itemize}
This chapter is relatively short (2-4 pages) and must leave the reader very clear on what the project is about and what your goals are.
\section{Motivation}
The design of a software system gradually evolves over time, as the understanding of the problem domain improves and the business requirements change. The cost of change of a software system is not negligible, however it is relatively low in comparison to other Engineering Disciplines \marginpar{I need to see if I can find a reference to back this up}and can be controlled by adhering to principles of good design and by following software development processes~\cite{Gof1995}. This propensity encourages iterative approach to software development; controlled improvements, extensions and revisions of the artefacts move the system from one version to the next(Agile). 

The key to controlled evolution of a software system is to rely on rigorous testing and validation at all levels of the architecture: from testing individual units of code (Unit Testing), through testing interfaces between components (Integration Testing), to testing a completely integrated software system (System Testing). Recent trends in software development practices even encourage the tests of a functionality to be written prior to the sourcecode~\cite{Beck2001}. It is argued that such systems tend to have low-coupling, as their design needs to accommodate testing components in isolation and control over the dependencies, however evidence is contradictory as to the exact effects of following this practice~\cite{Siniaalto2007}.

In order to ensure the correct operation of a software system over time, all the levels of testing need to be automatic and fast, in order to encourage the developers to perform them locally, following every change (Automated Testing, Continuous Integration). Adhering to those principles reduces reliance on manual testing, prevents regression errors in existing functionality (Regression Testing) and improves release time.

These principles apply well to deterministic systems; testing comes down to verifying that a set of inputs will produce a particular set of outputs, because the operation of such a system can be approximated with a deterministic-state machine. The difficulty occurs when dealing with non-deterministic systems; testing involves verifying that a particular set of inputs will produce a correct set of outputs with a statistically significant probability.

Since mid-80s, financial markets have been undergoing a phase of extensive automation of the way order flows are handled. First revolution took place with the introduction of electronic markets with an open access to the limit order book, that replaced traditional, open-outcry markets. However, since mid-2000s, a second revolution is in the making: Trading Algorithms have been replacing human traders in the process of negotiating prices and reducing market impact. Those sophisticated real-time systems, designed to replicate certain trading patterns, have a clear advantage over human traders in the amount of information they can take into account, as well as the speed at which they are able to take positions~\cite{Lenglet}.

The actions of the Trading Algorithms take an active part in shaping the market dynamics and, although constantly monitored, they sometimes contribute to \emph{flash crashes}: brief periods of extreme market volatility. On May 6, 2010, U.S. stock market has experienced a sudden price drop of 5\%, followed by a rapid recovery, all in the course of about 30 minutes. One subsequent analysis of that incident concluded that:
\begin{quote}
[$\ldots$] technological innovation is critical for market development. However, as markets change, appropriate safeguards must be implemented to keep pace with trading practices enabled by advances in technology~\cite{Kirilenko2011}.
\end{quote}
Overall, lack of rigorous testing of Trading Algorithms and other trading systems does not only affect the P\&L  of the owner of the technology, but may also lead to systemic instability of the market.

Trading Algorithms are an example of a non-deterministic system that operates in a highly non-deterministic environment. They are designed to respond to the situation on the market and to have a certain expectation as to the effect their actions will have on the market.

Traditionally, \marginpar{'Traditionally' is probably not the best word}Trading Algorithms are backtested on historical data by rebuilding the past order flows on the limit order book. Certain characteristics can be measured (e.g. VWAP of the algorithm) and compared to corresponding benchmarks (e.g. VWAP of the Market) to evaluate whether the algorithm is doing the right thing. This approach can give certain level of confidence as to the correct operation of the trading algorithm, however it nonetheless fails to model the reactions of other market participants to the order flows coming from the algorithm~\cite{Coggins2006}.

In recent years, the Agent-Based Modelling (ABM) approach has been applied to understanding the complex phenomena observed in economic and financial systems. An Agent-Based Model is a computer simulation of evolving and autonomous software decision makers (agents) that interact through a set of prescribed rules. Applying ABM models to financial markets offers possibility of studying how behaviour of individual agents affects the overall market dynamics~\citep{Sorban2008, Farmer2009}. Most widely discussed approaches to Agent-Based Artificial Stock Markets implement discrete time, call markets and employ various techniques in order to simulate traders' behaviour~\cite{Jha2010}.

In an attempt to supplement the strategy of backtesting Trading Algorithms on historical data and address its shortcomings, in this report we propose \emph{Eugene}: a real-time, continuous trading session Agent-Based Artificial Stock Market for Validation and System Testing of Trading Algorithms. We aim to demonstrate that simulations involving simple market participants (Noise Traders) can be used for discovering a class of logical polarity programming errors and differentiating between Trading Algorithms.

We will start by formulating the research objectives and null-hypotheses, followed by a discussion of the research methodologies.





\section{Research Objective and Null Hypotheses}
\section{Research methodology}
\section{Outline of the Report \label{Chapters/Introduction/Outline}}

In \Cref{Chapters/Background/Market-Microstructure} we describe the model of the market mechanism we will use in the simulator. We describe the market architectures employed in the stock exchanges during continuous trading sessions, namely model of limit order book, and how it enables market participants to interact asynchronously.

In \Cref{Chapters/Background/Agent-Based-Modelling} we summarise several artificial stock markets from the available literature. We provide a model of zero-intelligence agents that will be implemented by the Noise Traders, that is based on the work by~\citet[chap.~4]{Gilles2006}. Moreover, in order to provide a background for implementing a VWAP Trading Algorithm for the experiments, we survey available sources, most prominently the work in~\cite{Coggins2006, Kakade2004}. 

\Cref{Chapters/Analysis-and-Design} and \ref{Chapters/Implementation} provide a summary of the main requirements that informed the design and highlight main principles that guided the implementation of the simulator. We describe main elements of an Algorithmic Trading System and \textit{Eugene's} place in that architecture. We provide the pseudo code for the algorithms implemented for the Noise Traders and VWAP Traders.

\Cref{Chapters/Testing-and-Validation} summarises the steps taken to evaluate the implementation of the simulation, as well as it's efficacy in Testing Trading Algorithms. The results and analysis of experiments described in \Cref{Chapters/Implementation} are presented. Different levels of testing as they apply to the simulator are explained, from \textit{Unit-Testing} to following the method employed in~\cite[chap.~4]{Gilles2006} in order to validate the implementation of the Noise Traders.

Finally, in \Cref{Chapters/Conclusion-and-Further-Work} we summarise the key points of this work, highlight the testing method developed and its applicability to testing Trading Algorithms. We then review the contributions and achievements and then provide the possible extensions on how to take this work forward.





\chapter{Background}
\section{Market microstructure}
\label{Chapters/Background/Market-Microstructure}

In this section we describe the model of the market mechanism we will use in the simulator. We describe the market microstructure employed in major stock exchanges during continuous trading sessions, namely model of limit order book, and how it enables market participants to interact asynchronously (for a study of market microstructures in main stock exchanges, see~\cite{Comerton2004}). 

\subsection{Order Types}
Market participants indicate their willingness to trade in the form of trading instructions called \textit{orders}. We will consider only two types of orders: \textit{limit} and \textit{market} orders. A \textit{market order} specifies the instrument to trade, the quantity and the side of the trade (buy or sell). A \textit{limit order} additionally specifies a \textit{limit price}: maximum (buy) or minimum (sell) price that the trader accepts for an order. 

Traders can also cancel existing orders that have not been executed. \citet{Lilo2004} estimate that on the London Stock Exchange (LSE) on-book market, up to 30\% of outstanding limit orders are cancelled before execution and order cancellations play an important role in price formation mechanism. 

\subsection{The limit order book}
The limit order book is the leading market mechanism used by main stock exchanges during continuous trading. It consists of limit orders, characterised by a limit price, a volume and a time of arrival. The limit orders are sorted by price and time or arrival and stored in two queues: one for bid (buy) orders and one for ask (sell) orders. At a specific instant in time $t$, the order book can be described as~\cite{Gilles2006}: 
\begin{equation*}
\beta_n \leq \ldots \leq \beta_2 \leq \beta_1 < \alpha_1 \leq \alpha_2 \leq \ldots \alpha_m
\end{equation*}
where $\beta_i$ represent bid orders, $\alpha_j$ represent ask orders. The highest bid $\beta_1$ (or best bid) and lowest ask $\alpha_1$ (or best ask) define the spread $\alpha_1 - \beta_1$.

An incoming limit order can either trigger a trade or be stored in the book.    $\beta_1$ will be executed only if the book receives a market sell order, or a limit sell order with an offering price lower than or equal to $\beta_1$. In this case a trade will be executed at the price of $\beta_1$. Analogously for $\alpha_1$.







\section{Agent-Based Artificial Stock Markets}
\label{Chapters/Background/Agent-Based-Modelling}

In this section we summarise several artificial stock markets from the available literature. We provide a model of zero-intelligence agents that will be implemented by the Noise Traders, that is based on the work by~\citet[chap.~4]{Gilles2006}. Moreover, in order to provide a background material for implementing a VWAP Trading Algorithm for the experiments, we survey available sources, most prominently the work in~\cite{Coggins2006, Kakade2004}. 

\subsection{Agent-Based Modelling}
Agent-Based Modelling is a simulation technique concerned with designing societies of rule-based software agents that interact in particular ways, with a view of assessing the of of individual (or groups of) agents on the system as a whole. This technique has been successfully applied in many business scenarios, including financial simulations. Among the benefits of agent-based modelling is emergent phenomena: behaviour resulting from complex interactions of many individual entities.

\subsection{Agent-Based Computational Economics}
Agent-Based Models that attempt to explain economic processes are branded as Agent-Based Computational Economics. In recent years, studying stock markets using multi-agent based models has become a promising research area due to the fact that this methodology reflects the fundamental nature of a stock market, where the current situation is a result of a complex interaction of actions of many heterogenous investors that have various expectations and different levels of rationality.

\subsection{An overview of Agent-Based Artificial Stock Markets}

\subsubsection*{SantaFe ASM~\citep{Lebaron2002, Lebaron99}}
The Santa Fe Artificial Stock Market is a discrete time Artificial Stock Market that consists a central computational market and a number of intelligent agent. Agents make decision by attempting to forecast the future returns on the stock using genetic programming and therefore decide between investing in stock or leaving their money in the bank, which pays a fixed interest rate.

The simulated market consists of $N$ agents, usually $50-100$, that interact with the market. The stock has the price $p(t)$ per share at time $t$, where $p(t)$ is set endogenously to clear the market (\textit{call market}). The stock pays a dividend $d(t+1)$ at the end of time $t$, according to a stochastic process, independent of the market and agents' actions. The money left in the bank pays constant rate of return of $r$ per period. The information available to agents consists of the price, the dividend, total number of bids/asks at each past period, plus some additional variables. 

By using genetic programming, agents can explore a wide range of possible forecasting rules and they have the flexibility to use or disregard certain pieces of avaiable information. The market was able to generate the key stylised facts: weak forecastability, volatility persistence, and higher expected returns.~\cite{Lebaron99}. 

\subsubsection{Genoa ASM~\citep{}}









\section{VWAP Trading}


\chapter{Analysis and Design}
\label{Chapters/Analysis-and-Design}
In this Chapter we describe main elements of an Algorithmic Trading System and Eugene’s place in that architecture. We provide a summary of the main requirements, use cases and principles that informed the design of the simulator and the pseudo code for the algorithms implemented for the Noise Traders and VWAP Traders.
\section{Trading System Design} 

Various market participants submit orders to the stock exchange in order to trade. Different stock exchanges publish messages about the submitted/executed orders in a variety of formats. In order to deal with this complexity investment banks have internal market data buses that subscribe to feeds published by different stock exchanges in order to republish them in a standardised way to various internal systems, e.g. Trading Algorithms.

Similarly, in order to deal with complexity in submitting orders to different exchanges investment banks develop order management systems that register with various stock exchanges in order to provide a standardised way of submitting orders for various internal systems, e.g. Trading Algorithms, but also to allow Automatic Order Routing between different stock exchanges.

Historical data is stored in Tick Databases that subscribe to the market data bus in order to perform statistical analysis on the behaviour of different stocks. The result of the statistical analysis is a set of parameters which describe various aspects of stock behaviour. The parameters are published to various internal systems, e.g. Trading Algorithms.

Trading Algorithms are highly sophisticated and parameterised systems which accept parent orders and execute them using different strategies. Trading Algorithms register with the market data bus in order to react to the current situation on the market. They continuously compare the behaviour of stocks with their historical behaviour (using data and analysis from the tick database) in order to quickly discover unexpected behaviour. Having described the architecture of a trading system inside a bank, we can now turn to analyse how to approach testing the trading algorithms that operate within this architecture.

\section{Simulating a Trading System}

Trading Algorithms respond to messages received from the Stock Exchange (via the Market Data Bus) and send orders to the Stock Exchange (via the Order Management System). They expect the orders to have some influence on the situation on the Stock Exchange, based on the analysis of past behaviour received from the Tick Database. In our implementation we are going to explicitly omit the issue of past behaviour analysis and the Tick Database, as it is beyond the scope of this report. 

Therefore, in order to simulate a Trading System for testing Trading Algorithms, two systems need to be put in place:
\begin{itemize}
\item Order Matching Engine that will accept orders on to a limit order book, match orders and send executions/market data back to traders.
\item Market behaviour simulator to generate a realistic order flow. \end{itemize}

Having stated the problem we are trying to solve and the required architecture, we will now present a formal set of requirements that the system needs to fulfil.

\section{Requirements Analysis}
Having analysed available literature on the design of asynchronous, limit order book markets in \Cref{Chapters/Background/Market-Microstructure}, as well as existing Agent-Based ASMs in \Cref{Chapters/Background/Agent-Based-Modelling}, we will now list the requirements that the trading simulator needs to satisfy.

Requirements are prioritised according to the MoSCoW approach:

\begin{itemize}
\item M - Must - These requirements are the highest priority must be completed to ensure project success.
\item S - Should - These requirements are high priority and will be completed after the M requirements.
\item C - Could - These requirements will be worked on only if there is time after the M and S requirements or if they do not interfere with the progress of the higher priority tasks.
\item W - Would - These requirements most probably will not be done and are not necessary for the completion and success of the project but are good ideas for future work.
\end{itemize}

\subsection{Functional Requirements}
\begin{center}
\begin{longtable}[htbp]{c p{4.2in} c }

\multicolumn{1}{c}{\textbf{ID}}           &
\multicolumn{1}{c}{\textbf{Requirement}}  &
\multicolumn{1}{c}{\textbf{Priority}}     \\              
\toprule

\multicolumn{3}{c}{\textbf{Market Agent}}   	         \\
F01  & The Market Agent shall match orders on the limit order book & M \\ 
F02  & The Market Agent shall accept limit/market orders and order cancellations. & M \\
F03  & The Market Agent shall provide \textit{level 2} market data access with anonymous order ids. & M \\
F04  & The Market Agent shall report order executions and status changes back to the order owner. & M \\ 

\multicolumn{3}{c}{\textbf{Simulation Agent}}   \\
F05  & The Simulation Agent shall start all specified agents.  & M \\
F06  & The Simulation Agent shall synchronise the start of a simulation. & M \\
F07  & The Simulation Agent shall stop a simulation after a specified amount of time. & M \\
F09  & The Simulation Agent shall bootstrap the limit order book before starting the simulation. & M \\ 

\multicolumn{3}{c}{\textbf{Client API}}   \\
F10  & The Client API shall handle the low-level interaction with the Market Agent and Simulation Agent and provide callback methods for receiving messages. & M \\

\multicolumn{3}{c}{\textbf{Other}}   		             \\
F11  & The System shall log all events to a file.        & M \\
         
\end{longtable}
\end{center}

\needspace{11\baselineskip}
\subsection{Non Functional Requirements}

\marginpart{Is N02 a functional requirement?} 

\begin{table}[htbp]
\begin{center}
\begin{longtable}{c p{4.2in} c }

\multicolumn{1}{c}{\textbf{ID}}           &
\multicolumn{1}{c}{\textbf{Requirement}}  &
\multicolumn{1}{c}{\textbf{Priority}}     \\        
\toprule

N01  & The System shall handle incoming order rate of              & M \\
N02  & The System shall maintain $\geq70\%$ unit test coverage.    & M \\
N03  & The System shall be highly modular and decoupled.           & M \\
N04  & The System shall be integration tested.	                   & M \\
N05  & The System shall be $100\%$ javadoced.	                   & M \\
N06  & The System shall closely approximate message types and formats of industry standard messaging formats. & M \\
N07  & The System shall be implemented on top of JADE Framework.   & M \\
N08  & The System shall publish performance statistics at runtime. & C \\
 
\end{longtable}
\end{center}
\end{table}


\FloatBarrier
\section{Use case analysis}


\newcounter{usecases}
\addtocounter{usecases}{1}

\newenvironment{usecase}[1]
  {
   \subsection{UC\arabic{usecases}: {#1}}%
   \stepcounter{usecases}%
   \begin{samepage}%
   \begin{list}{}%
   {\topsep=5em
   \renewcommand\makelabel[1]{\textbf{##1}\hfill}%
   \settowidth\labelwidth{\makelabel{Post-Conditions} + 0.5in}%
   \setlength\leftmargin{\labelwidth}
   \addtolength\leftmargin{\labelsep}}}
  {\end{list}%
   \end{samepage}}

% UC1   
\begin{usecase}{Start the simulation}
\item[Primary Actors] Simulation Agent
\item[Secondary Actors] Market Agent, Trader Agents
\item[Description] The Simulation Agent starts the simulation.
\item[Pre-conditions] 
\item[Flow of Events] 
\begin{enumerate}
\item The Simulation Agent starts the Market Agent.
\item The Market Agent starts and sends a Started message back to the Simulation Agent.
\item The Simulation Agent receives the Started message.
\item The Simulation Agent starts the Trader Agents.
\item The Trader Agents start and log on with the Market Agent.
\item The Trader Agents send a Logon Complete message back to the Simulation Agent.
\item The Simulation Agent receives the Logon Complete messages.
\item After receiving all the Logon Complete messages, the Simulation Agent sends initial limit orders to the Market Agent.
\item After receiving acknowledgements for the initial limit orders, the Simulation Agent sends Start messages to the Trader Agents.
\item The Trader Agents send Started messages back to the Simulation Agent.
\item The Simulation Agent receives Started messages.
\item After receiving all Started messages, the Simulation Agent sleeps until the end of the simulation.
\end{enumerate}
\item[Post-conditions] The Simulation Agent sleeps.
\item[Alternative Flows] None. 
\end{usecase}
  
% UC2
\begin{usecase}{Logon}
\item[Primary Actors] Trader Agent
\item[Secondary Actors] Market Agent
\item[Description] The Trader Agent sends a Logon message to the Market Agent.
\item[Pre-conditions] 
\item[Flow of Events] 
\begin{enumerate}
\item The Trader Agent sends a Logon message to the Market Agent
\item The Market Agent receives the message and stores the Trader Agent ID.
\item The Market Agent sends an acknowledgement message back to the Trader Agent.
\item The Trader Agent receives the acknowledgement message.
\end{enumerate}
\item[Post-conditions] The Trader Agent ID has been saved. \item[Alternative Flows] None.
\end{usecase}
  
% UC3 
\begin{usecase}{Send Limit Order}
\item[Primary Actors] Trader Agent
\item[Secondary Actors] Market Agent, Execution Engine
\item[Description] The Trader Agent sends a Limit Order message to the Market Agent.
\item[Pre-conditions] The Trader Agent has logged on with the Market Agent and received the Start message from the Simulation Agent.
\item[Flow of Events] 
\begin{enumerate}
\item The Trader Agent sends a New Limit Order message to the Market Agent
\item The Market Agent receives the message and forwards the limit order to the Execution Engine.
\item The Market Agent sends an acknowledgement message back to the Trader Agent.
\item The Trader Agent receives the acknowledgement message.
\end{enumerate}
\item[Post-conditions] The Execution Engine accepted the New Limit Order message.
\item[Alternative Flows] None
\end{usecase}

% UC4 
\begin{usecase}{Send Market Order}
\item[Primary Actors] Trader Agent
\item[Secondary Actors] Market Agent, Execution Engine
\item[Description] The Trader Agent sends a New Market Order message to the Market Agent.
\item[Pre-conditions] The Trader Agent has logged on with the Market Agent and received the Start message from the Simulation Agent.
\item[Flow of Events] 
\begin{enumerate}
\item The Trader Agent sends a New Market Order message to the Market Agent.
\item The Market Agent receives the message and forwards the market order to the Execution Engine.
\item The Market Agent sends an acknowledgement message back to the Trader Agent.
\item The Trader Agent receives the acknowledgement message.
\end{enumerate}
\item[Post-conditions] The Execution Engine accepted the New Market Order message.
\item[Alternative Flows] None
\end{usecase}

% UC5
\begin{usecase}{Cancel Order}
\item[Primary Actors] Trader Agent
\item[Secondary Actors] Market Agent, Execution Engine
\item[Description] The Trader Agent sends a Cancel Order message to the Market Agent.
\item[Pre-conditions] The Trader Agent has logged on with the Market Agent and received the Start message from the Simulation Agent.
\item[Flow of Events] 
\begin{enumerate}
\item The Trader Agent sends a Cancel Order message to the Market Agent.
\item The Market Agent receives the message, checks if the order exists and forwards the Cancel Order message to the Execution Engine.
\item The Market Agent sends an acknowledgement message back to the Trader Agent.
\item The Trader Agent receives the acknowledgement message.
\end{enumerate}
\item[Post-conditions] The Cancel Order message has been accepted by the Execution Engine.
\item[Alternative Flows] The order referred to in the Cancel Order message does not exist, therefore the Market Agent sends a Cancel Reject message back to the Trader Agent.
\end{usecase}

% UC6
\begin{usecase}{Process New Limit Order message}
\item[Primary Actors] Execution Engine
\item[Secondary Actors] Trader Agents
\item[Description] The Execution Engine processes a New Limit Order message.
\item[Pre-conditions] A New Limit Order message has been accepted by the Execution Engine.
\item[Flow of Events] 
\begin{enumerate}
\item The Execution Engine checks if there are matching limit orders on the opposite side of the book of the limit order to execute.
\item The Execution Engine executes all matching limit orders and sends Execution Report messages to the counter-parties and Order Executed messages to the rest of the Trader Agents.
\item If the limit order has not been filled, the Execution Engine puts the limit order into the limit order book and sends Add Order messages to all Trader Agents.
\end{enumerate}
\item[Post-conditions] The matching limit orders have been executed, appropriate messages have been sent and the remaining quantity has been put into the limit order book.
\item[Alternative Flows] None.
\end{usecase}

% UC7
\begin{usecase}{Process New Market Order message}
\item[Primary Actors] Execution Engine
\item[Secondary Actors] Trader Agents
\item[Description] The Execution Engine processes a New Market Order message.
\item[Pre-conditions] The Execution Engine accepted a New Market Order message.
\item[Flow of Events] 
\begin{enumerate}
\item The Execution Engine checks if there are matching limit orders on the opposite side of the book of the market order to execute.
\item The Execution Engine executes all matching limit orders and sends Execution Report messages to the counter-parties and Order Executed messages for the limit orders to the rest of the Trader Agents.
\item If the market order has not been filled, the Execution Engine sends a Order Canceled message for the remaining quantity back to the Trader Agent that sent the market order.
\end{enumerate}
\item[Post-conditions] The market order has been executed, matching limit orders have been executed, appropriate messages have been sent and the remaining quantity has been canceled.
\item[Alternative Flows] None.
\end{usecase}

% UC8   
\begin{usecase}{Process Cancel Order message}
\item[Primary Actors] Execution Engine
\item[Secondary Actors] Trader Agents
\item[Description] The Execution Engine processes a Cancel Order message.
\item[Pre-conditions] The Execution Engine accepted a Cancel Order message. 
\item[Flow of Events] 
\begin{enumerate}
\item The Execution Engine removes the limit order from the limit order book.
\item The Execution Engine sends an Order Canceled message for the remaining quantity back to the Trader agent that sent the limit order.
\item The Execution Engine sends a Delete Order message to all the Trader Agents.
\end{enumerate}
\item[Post-conditions] The limit order has been canceled.
\item[Alternative Flows] None.
\end{usecase}

% UC9   
\begin{usecase}{Stop the simulation}
\item[Primary Actors] Simulation Agent
\item[Secondary Actors] Trader Agents
\item[Description] The Simulation Agent stops the simulation.
\item[Pre-conditions] The simulation is running. 
\item[Flow of Events] 
\begin{enumerate}
\item The Simulation Agent wakes up.
\item The Simulation Agent sends a Stop message to the Trader Agents.
\item The Trader Agents stop and send Stopped messages back to the Simulation Agent.
\item The Simulation Agent receives the Stopped messages.
\item After receiving all the Stopped messages, the Simulation Agent stops.
\end{enumerate}
\item[Post-conditions] The simulation is stopped.
\item[Alternative Flows] 
\end{usecase}


\chapter{Implementation and Testing}
\label{Chapters/Implementation}
\section{Overall Principles}
In this section I will describe the overall programming and design principles that have informed the implementation of this system in order to later refer to them in the later sections that describe the particular modules of the system. The principles will be described in no particular order, as their scope usually cuts through different architectural levels.

\paragraph{Design by contract and assertions} 
\marginpar{Replace with an example from the codebase}
Input is checked for correctness and the contract for the method or constructor parameters is clearly documented. To that end, \texttt{com.google.common.base.Preconditions} class from \texttt{guava-libraries}\cite{guava} is used extensively. Example of a constructor that checks the parameter for \texttt{null reference} and asserts that its length is not zero:
\lstinputlisting[caption=Checking for \texttt{null reference} and length of the \textit{String} parameter]{Code/Implementation/Overall-Principles/checkNotNullExample.java}

This method would have corresponding tests that check whether the correctness of parameters is correctly verified:
\lstinputlisting[caption=The corresponding unit tests]{Code/Implementation/Overall-Principles/checkNotNullTestExample.java}

Similarly, post-conditions and invariants would be verified. Strict adherence to these principles allows for much safer code that is guaranteed to be used correctly.

\paragraph{Separation of concerns, programming to an interface and information hiding}
The code is separated into many modules, so that responsibility of each part of the code is very strictly defined. The APIs between different modules are defined in terms of interfaces and specific implementations are hidden behind factories and private packages. The following is an example from the \textit{market/book} module:
\lstinputlisting[caption=market/book/src/main/java/eugene/market/book/OrderBook.java]{Code/Implementation/Overall-Principles/OrderBook.java}
\lstinputlisting[caption=market/book/src/main/java/eugene/market/book/OrderBooks.java]{Code/Implementation/Overall-Principles/OrderBooks.java}

\paragraph{Immutable objects}
An object is considered immutable if its state cannot be changed after it is constructed. Reliance on immutable objects is considered a sound strategy for creating simple code. Immutable objects have an additional advantage of being naturally \textit{thread-safe}: since there is no state that can be changed, they can be safely shared by multiple threads, without interference. The following is an example from the \textit{market/book} module:
\lstinputlisting[caption=market/book/src/main/java/eugene/market/book/OrderBooks.java]{Code/Implementation/Overall-Principles/Order.java}


\begin{itemize}
\item Extensive unit and integration testing.
\item mocking.
\item Careful consideration of threading issues and hence multithreaded design.


\end{itemize}

\section{Overall Architecture}
Before I describe the architecture of all the modules that make up \textit{eugene}, I will present the main technologies used in the implementation.

\subsection{Java Agent DEvelopment Framework}
\texttt{JADE} is a software framework for developing distributed, multi-agent software systems in Java. \texttt{JADE} can be deployed onto a set of containers (nodes) which together form a platform (cluster).

Each platform has a \texttt{Main Container} which holds two special agents:
\begin{itemize}
\item The \texttt{Agent Management System} (AMS) which can create and kill agents, kill containers and shut down the entire platform.
\item The \texttt{Directory Facilitator} (DF) which implements a \texttt{yellow pages} service.
\end{itemize} 

Each agent in \texttt{JADE} operates in a separate thread of control, therefore allowing independent, preemptive behaviour. \texttt{JADE}  has been designed to operate within an \texttt{OSGi} container, therefore making it easier to deploy.

\subsection{Open Services Gateway Initiative Framework}
\texttt{OSGi} is dynamic module and service system platform for Java. Application components (distributed as \texttt{bundles}) can be remotely managed without requiring a reboot of the entire container. A package management system enables developers to package public and private APIs within the same bundle, but only expose public APIs at runtime. The dynamic service system allows the components to discover the addition of new services and act accordingly.

\subsection{Simple Logging Facade for Java}
\texttt{SLF4J} is an abstraction for various logging frameworks that allows the user to plug in the desired implementation at deployment time. Combined with \texttt{logback} implementation, it is a very powerful and versatile logging framework. Among the most useful features are \textit{Markers}: ability to tag a log entry to indicate the type of the event being logged. Different \texttt{markers} can then be directed to separate files.

\section{jade-unit}
In order to fulfil requirement \textit{NF04}, there needs to be a way to create and tear down an arbitrary number of \texttt{JADE} containers, potentially within a single JVM. Being able to do that would allow for reuse of existing testing tools and frameworks. 

There exists a \texttt{Test Suite Framework} available through the official \texttt{JADE} website, however it seemed like maintaining two entirely different testing environments would be an unnecessary complication.

Unfortunately, the \textit{JADE} version as available on the official website contains a messaging module for sending messages to external nodes that opens a socket, and renders creating multiple \texttt{JADE} containers impossible.

Luckily, \texttt{JADE} architecture is modular and hence the implementation of the messaging module can be replaced. \texttt{jade-unit} extension implements a no-op messaging module that simply ignores any messages directed to it. This allows the integration tests to reuse the entire testing architecture employed for unit tests. It is expected that the patch file with \texttt{jade-unit} will be forwarded to the \texttt{JADE} maintainers, in due time.

\section{Order Book}
\section{Simulation Ontology (simulation/ontology)}
This module defines message types necessary in order to implement use case \textit{UC1}. All message types and fields are listed \Cref{Appendix/Simulation-Ontology}.
\section{Simulation Agent (simulation/agent/)}
The \texttt{Simulation Agent} implements use case \textit{UC1}. It starts the \texttt{Market Agent}, awaits for messages that synchronise the start of \texttt{Trader Agents}, bootstraps the order book and starts and stops the simulation.
\chapter{Market Ontology Messages}
\label{Appendix/Market-Ontology}

\section{Fields (eugene.market.ontology.field)}

\FloatBarrier
\subsection{Single valued fields}

\begin{table}[htbp]
\begin{center}
\begin{tabular}{l l p{3.5in} l}

\multicolumn{1}{l}{\textbf{Tag}}            &
\multicolumn{1}{l}{\textbf{Field}}  	    &
\multicolumn{1}{l}{\textbf{Description}}    & 
\multicolumn{1}{l}{\textbf{Type}}		   	\\              
\toprule

6  	& AvgPx  	& Average price of all fills of an order. & BigDecimal \\

11 	& ClOrdID	& Unique identifier for an order as assigned by the Agent originating the trade. Uniqueness must be guaranteed within a single trading day. & String \\

14	& CumQty	& Total quantity (e.g. number of shares) filled. & Long \\

31  & LastPx	& Price of execution.	& BigDecimal \\

32  & LastQty	& Quantity executed.	& Long	     \\

151	& LeavesQty	& Quantity open for further execution. If the \texttt{OrdStatus} is \texttt{OrdStatus\#CANCELLED} or \texttt{OrdStatus\#REJECTED} (in which case the order is no longer active) then \texttt{LeavesQty} could be $0$, otherwise $\mbox{LeavesQty} = \mbox{OrderQty} - \mbox{CumQty}$. & Long \\

37	& OrderID	& Unique identifier for Order as assigned by the Market. Uniqueness must be guaranteed within a single trading day.  & String \\

38  & OrderQty	& Quantity ordered.	& Long \\

44	& Price 	& Price per unit of quantity.	& BigDecimal  \\

55	& Symbol 	& Ticker symbol. Common, "human understood" representation of the security.	& String \\

17	& TradeID	& The unique ID assigned to the execution by the Market Agent. & String \\

\end{tabular}
\end{center}
\caption{\texttt{Market Ontology} single valued fields.}        
\end{table}

\pagebreak
\FloatBarrier
\subsection{Enum fields}
\begin{table}[ht]
\begin{center}
\begin{tabular}{l l p{3in} p{1.5in}}

\multicolumn{1}{l}{\textbf{Tag}}            &
\multicolumn{1}{l}{\textbf{Field}}  	    &
\multicolumn{1}{l}{\textbf{Description}}    & 
\multicolumn{1}{l}{\textbf{Values}}			\\              
\toprule

150  & ExecType  & Describes the specific \texttt{ExecutionReport}'s status. & 
NEW, CANCELED, REJECTED, TRADE \\

39	 & OrdStatus & Identifies current status of order. & NEW, PARTIALLY\_FILLED, FILLED, CANCELLED, REJECTED \\

40	 & OrdType	& Order type.	& MARKET, LIMIT	\\

1409 & SessionStatus	& Session status at time of \texttt{Logon}. Field is intended to be used when the \texttt{Logon} is sent as an acknowledgement from acceptor of the \texttt{Logon} message.	& SESSION\_ACTIVE  \\

54	& Side		& Side of order.	& BUY, SELL \\
\end{tabular}
\end{center}
\caption{\texttt{Market Ontology} enum fields.}        
\end{table}

\section{Messages}
\FloatBarrier
\subsection{Order management messages (eugene.market.ontology.message)}
\begin{table}[htbp]
\begin{center}
\begin{tabular}{l p{1.2in} p{2.8in} p{1.5in}}

\multicolumn{1}{l}{\textbf{Type}}            &
\multicolumn{1}{l}{\textbf{Message}}  	    &
\multicolumn{1}{l}{\textbf{Description}}    & 
\multicolumn{1}{l}{\textbf{Fields [optional]}}	\\              
\toprule

8  & \footnotesize{ExecutionReport} & 
The \texttt{ExecutionReport} message is used to:
\begin{itemize}
\item Confirm the receipt of an order.
\item Confirm changes to an existing order (i.e. accept cancel request).
\item Relay fill information on working orders.
\item Reject orders.
\end{itemize} &
\texttt{ExecType}, \texttt{Symbol}, \texttt{Side}, \texttt{LeavesQty}, \texttt{CumQty}, \texttt{AvgPx[Yes]}, \texttt{OrdStatus}, \texttt{OrderID}, \texttt{ClOrdID}, \texttt{LastPx[Yes]}, \texttt{LastQty[Yes]}  \\

A	& Logon & 
First message sent by the Agent to the Market in order to register with the Market.
\begin{itemize}
\item As a result, the Agent will receive data feed updates.
\item If logon was successful, the Market will send back the same \texttt{Logon} message with \texttt{Logon\#SessionStatus} equal to \texttt{SessionStatus\#SESSION\_ACTIVE}, otherwise this field will be \texttt{null}. 
\end{itemize} &
\texttt{Symbol}, \texttt{SessionStatus[Yes]}  \\

D	& \footnotesize{NewOrderSingle} & Used by agents wishing to submit securities orders to the Market for execution. &
\texttt{ClOrdID}, \texttt{Symbol}, \texttt{Side}, \texttt{OrderQty}, \texttt{OrdType}, \texttt{Price[Yes]} \\

9	& \footnotesize{OrderCancelReject}	& Issued by the Market upon receipt of a \texttt{OrderCancelRequest} message which cannot be honored. & 
\texttt{OrderID}, \texttt{ClOrdID}, \texttt{OrdStatus} \\

F	& \footnotesize{OrderCancelRequest}	& Requests the cancellation of all of the remaining quantity of an existing order. &
\texttt{ClOrdID}, \texttt{Symbol}, \texttt{Side}, \texttt{OrderQty} \\

\end{tabular}
\end{center}
\caption{Order management messages.}        
\end{table}
\clearpage

\Floatbarrier
\subsection{Market data messages (eugene.market.ontology.message.data)}
\begin{table}[ht]
\begin{center}
\begin{tabular}{l p{1in} p{3in} p{1.5in}}

\multicolumn{1}{l}{\textbf{Type}}            &
\multicolumn{1}{l}{\textbf{Message}}  	    &
\multicolumn{1}{l}{\textbf{Description}}    & 
\multicolumn{1}{l}{\textbf{Fields [optional]}}	\\              
\toprule

BP0x21	& AddOrder & Represents a newly accepted visible order on the order book. & \texttt{OrderID}, \texttt{Symbol}, \texttt{Side}, \texttt{OrderQty}, \texttt{Price}  \\

BP0x29	& DeleteOrder	& Sent whenever an open order is completely cancelled. The \texttt{OrderID} refers to the \texttt{OrderID} of the original \texttt{AddOrder} message. & \texttt{OrderID} \\

BP0x23	& OrderExecuted	& Sent when a visible order on the order book is executed in whole or in part at a price. The \texttt{OrderID} refers to the \texttt{OrderID} of the original \texttt{AddOrder} message. & \texttt{OrderID}, \texttt{LastQty}, \texttt{LeavesQty}, \texttt{LastPx}, \texttt{TradeID} \\

\end{tabular}
\end{center}
\caption{Market data messages.}        
\end{table}

\section{Market Agent}
\section{Client API (market/client/)}
\label{Chapters/Background/Client-API}
The \texttt{Client API} fulfils requirement \textit{F10}. The primary purpose is to handle the intricacies of the simulation startup, provide  a single gateway responsible for receiving messages and routing them to the user code, and provide simple method calls that construct \texttt{JADE} messages from \texttt{Market Ontology} objects.

Following the principles outlined in \Cref{Chapters/Implementation/Overall-Principles}, an interface is defined in \\ \texttt{eugene.market.client.Session} that specifies the methods for sending   and extracting \texttt{Market Ontology} objects from \texttt{JADE} messages. The implementation is located in \\\texttt{eugene.market.client.impl.SessionImpl}, however no client code uses this class directly. In order to connect to a \texttt{Market Agent} defined by the current \texttt{Simulation}, the \texttt{Trader Agent} will obtain a \texttt{Behaviour} from \texttt{eugene.market.client.Sessions} factory that will establish the connection. The actual implementation is located in \texttt{eugene.market.client.impl.SessionInitiator}, in a private package. 

In order to establish a \texttt{Session}, the \texttt{Trader Agent} needs to provide an implementation of the \texttt{eugene.market.client.Application} interface, that will receive notifications of all messages that will be sent and received by the \texttt{Session}. If the \texttt{Trader Agent} requires several objects listening to messages, those can be routed using an implementation of \texttt{eugene.market.client.ProxyApplication}, obtained from \\\texttt{eugene.market.client.Applications} factory. \texttt{ProxyApplication} will route messages to other \texttt{Application} instances given to it (\texttt{Application} instances can be added or removed dynamically).

One use case, when the \texttt{Trader Agent} requires messages to be routed to several objects, is when there is a need to build the order book, as well as calculate statistics about the behaviour of the instrument traded. In the interest of separation of concerns, these two roles should be implemented by different classes, and indeed such separation is possible thanks to the \texttt{ProxyApplication}.

In the interest of code reuse, an instance of an \texttt{Application} that will build the order book can be obtained from the \texttt{Applications} factory. Th \texttt{proxy} pattern allows for ease of composing different \texttt{Application} instances to create complex dependency graphs: indeed an example is \texttt{eugene.market.client.TopOfBookApplication} that requires an instance of \texttt{OrderBook} that is updated from messages and exposes a simple API for tracking the last price at the top of either side of the book.

Last functionality worth mentioning is that \texttt{Session} tracks the status of orders submitted: user code can register an instance of \\ \texttt{eugene.market.client.OrderReferenceListener} to receive updates about a single order. Similarly to \texttt{Application}, order updates can be routed if more than one instance of \texttt{OrderReferenceListener} needs to be notified \\ (\texttt{eugene.market.client.OrderReferenceListenerProxy} that can be obtained from \texttt{eugene.market.client.OrderReferenceListeners} factory).

\chapter{Experiments and Results}
\label{Chapters/Testing-and-Validation}

\begin{itemize}
\item Describe your testing strategy (unit, functional, acceptance testing
and how they are carried out). How were test cases selected.
\item Examples of specific tests and how they were carried out (e.g., using mock objects to break dependencies). Focus on the interesting cases.
\item A summary of the test results and what coverage was achieved. The detailed test report(s) should appear in the appendix.
\end{itemize}

If your project requires substantial evaluation of data and results, or other forms of testing that are not code-based, then adapt this chapter to suit.
This chapter will typically be 2-4 pages in length but could be more de- pending on the depth of testing done.

\chapter{Conclusions and Further Work}
\label{Chapters/Conclusion-and-Further-Work}

\begin{itemize}
\item Summary.
\item Conclusions.
\item How to use it and take it further.
\end{itemize}


\begin{itemize}
\item A summary of what the project has achieved. Address each goal set
out in the introduction.
\item A critical evaluation of the results of the project (e.g., how well were the goals met, is the application fit for purpose, has good design and implementation practice been followed, was the right implementation technology chosen and so on).
\item Future work. How could the project be developed if you had another 6 months.
\item Wrap-up and final thoughts on your project.
\end{itemize}

This chapter is typically 2-4 pages long but could be longer if the project
work requires more extensive evaluation.

\appendix
\section{Simulation Ontology (simulation/ontology)}
This module defines message types necessary in order to implement use case \textit{UC1}. All message types and fields are listed \Cref{Appendix/Simulation-Ontology}.
\chapter{Market Ontology Messages}
\label{Appendix/Market-Ontology}

\section{Fields (eugene.market.ontology.field)}

\FloatBarrier
\subsection{Single valued fields}

\begin{table}[htbp]
\begin{center}
\begin{tabular}{l l p{3.5in} l}

\multicolumn{1}{l}{\textbf{Tag}}            &
\multicolumn{1}{l}{\textbf{Field}}  	    &
\multicolumn{1}{l}{\textbf{Description}}    & 
\multicolumn{1}{l}{\textbf{Type}}		   	\\              
\toprule

6  	& AvgPx  	& Average price of all fills of an order. & BigDecimal \\

11 	& ClOrdID	& Unique identifier for an order as assigned by the Agent originating the trade. Uniqueness must be guaranteed within a single trading day. & String \\

14	& CumQty	& Total quantity (e.g. number of shares) filled. & Long \\

31  & LastPx	& Price of execution.	& BigDecimal \\

32  & LastQty	& Quantity executed.	& Long	     \\

151	& LeavesQty	& Quantity open for further execution. If the \texttt{OrdStatus} is \texttt{OrdStatus\#CANCELLED} or \texttt{OrdStatus\#REJECTED} (in which case the order is no longer active) then \texttt{LeavesQty} could be $0$, otherwise $\mbox{LeavesQty} = \mbox{OrderQty} - \mbox{CumQty}$. & Long \\

37	& OrderID	& Unique identifier for Order as assigned by the Market. Uniqueness must be guaranteed within a single trading day.  & String \\

38  & OrderQty	& Quantity ordered.	& Long \\

44	& Price 	& Price per unit of quantity.	& BigDecimal  \\

55	& Symbol 	& Ticker symbol. Common, "human understood" representation of the security.	& String \\

17	& TradeID	& The unique ID assigned to the execution by the Market Agent. & String \\

\end{tabular}
\end{center}
\caption{\texttt{Market Ontology} single valued fields.}        
\end{table}

\pagebreak
\FloatBarrier
\subsection{Enum fields}
\begin{table}[ht]
\begin{center}
\begin{tabular}{l l p{3in} p{1.5in}}

\multicolumn{1}{l}{\textbf{Tag}}            &
\multicolumn{1}{l}{\textbf{Field}}  	    &
\multicolumn{1}{l}{\textbf{Description}}    & 
\multicolumn{1}{l}{\textbf{Values}}			\\              
\toprule

150  & ExecType  & Describes the specific \texttt{ExecutionReport}'s status. & 
NEW, CANCELED, REJECTED, TRADE \\

39	 & OrdStatus & Identifies current status of order. & NEW, PARTIALLY\_FILLED, FILLED, CANCELLED, REJECTED \\

40	 & OrdType	& Order type.	& MARKET, LIMIT	\\

1409 & SessionStatus	& Session status at time of \texttt{Logon}. Field is intended to be used when the \texttt{Logon} is sent as an acknowledgement from acceptor of the \texttt{Logon} message.	& SESSION\_ACTIVE  \\

54	& Side		& Side of order.	& BUY, SELL \\
\end{tabular}
\end{center}
\caption{\texttt{Market Ontology} enum fields.}        
\end{table}

\section{Messages}
\FloatBarrier
\subsection{Order management messages (eugene.market.ontology.message)}
\begin{table}[htbp]
\begin{center}
\begin{tabular}{l p{1.2in} p{2.8in} p{1.5in}}

\multicolumn{1}{l}{\textbf{Type}}            &
\multicolumn{1}{l}{\textbf{Message}}  	    &
\multicolumn{1}{l}{\textbf{Description}}    & 
\multicolumn{1}{l}{\textbf{Fields [optional]}}	\\              
\toprule

8  & \footnotesize{ExecutionReport} & 
The \texttt{ExecutionReport} message is used to:
\begin{itemize}
\item Confirm the receipt of an order.
\item Confirm changes to an existing order (i.e. accept cancel request).
\item Relay fill information on working orders.
\item Reject orders.
\end{itemize} &
\texttt{ExecType}, \texttt{Symbol}, \texttt{Side}, \texttt{LeavesQty}, \texttt{CumQty}, \texttt{AvgPx[Yes]}, \texttt{OrdStatus}, \texttt{OrderID}, \texttt{ClOrdID}, \texttt{LastPx[Yes]}, \texttt{LastQty[Yes]}  \\

A	& Logon & 
First message sent by the Agent to the Market in order to register with the Market.
\begin{itemize}
\item As a result, the Agent will receive data feed updates.
\item If logon was successful, the Market will send back the same \texttt{Logon} message with \texttt{Logon\#SessionStatus} equal to \texttt{SessionStatus\#SESSION\_ACTIVE}, otherwise this field will be \texttt{null}. 
\end{itemize} &
\texttt{Symbol}, \texttt{SessionStatus[Yes]}  \\

D	& \footnotesize{NewOrderSingle} & Used by agents wishing to submit securities orders to the Market for execution. &
\texttt{ClOrdID}, \texttt{Symbol}, \texttt{Side}, \texttt{OrderQty}, \texttt{OrdType}, \texttt{Price[Yes]} \\

9	& \footnotesize{OrderCancelReject}	& Issued by the Market upon receipt of a \texttt{OrderCancelRequest} message which cannot be honored. & 
\texttt{OrderID}, \texttt{ClOrdID}, \texttt{OrdStatus} \\

F	& \footnotesize{OrderCancelRequest}	& Requests the cancellation of all of the remaining quantity of an existing order. &
\texttt{ClOrdID}, \texttt{Symbol}, \texttt{Side}, \texttt{OrderQty} \\

\end{tabular}
\end{center}
\caption{Order management messages.}        
\end{table}
\clearpage

\Floatbarrier
\subsection{Market data messages (eugene.market.ontology.message.data)}
\begin{table}[ht]
\begin{center}
\begin{tabular}{l p{1in} p{3in} p{1.5in}}

\multicolumn{1}{l}{\textbf{Type}}            &
\multicolumn{1}{l}{\textbf{Message}}  	    &
\multicolumn{1}{l}{\textbf{Description}}    & 
\multicolumn{1}{l}{\textbf{Fields [optional]}}	\\              
\toprule

BP0x21	& AddOrder & Represents a newly accepted visible order on the order book. & \texttt{OrderID}, \texttt{Symbol}, \texttt{Side}, \texttt{OrderQty}, \texttt{Price}  \\

BP0x29	& DeleteOrder	& Sent whenever an open order is completely cancelled. The \texttt{OrderID} refers to the \texttt{OrderID} of the original \texttt{AddOrder} message. & \texttt{OrderID} \\

BP0x23	& OrderExecuted	& Sent when a visible order on the order book is executed in whole or in part at a price. The \texttt{OrderID} refers to the \texttt{OrderID} of the original \texttt{AddOrder} message. & \texttt{OrderID}, \texttt{LastQty}, \texttt{LeavesQty}, \texttt{LastPx}, \texttt{TradeID} \\

\end{tabular}
\end{center}
\caption{Market data messages.}        
\end{table}

\chapter{Appendix}

\begin{itemize}
\item User manual - how to use it.
\item System manual = how to compile, make changes etc.
\item If overflowing, drop page numbers on Appendixes.
\item Have code examples that are being referred form Implementation.
\item Experiment results. 
\end{itemize}


%\end{onehalfspace}

%\begin{singlespace}
%\begin{footnotesize}
%\begin{twocolumn}
\bibliographystyle{plainnat}
%\bibliographystyle{IEEEtran}
\bibliography{Bibliography}
%\end{twocolumn}
%\end{footnotesize}
%\end{singlespace}

\end{document}
